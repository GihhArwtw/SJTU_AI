\documentclass{article}
\usepackage[utf8]{inputenc}
\usepackage{amsmath}
\usepackage{amsfonts}
\usepackage{setspace}
\usepackage{amsthm}
\usepackage{amssymb}
\usepackage{geometry}
\usepackage{verbatim}
\usepackage{mathrsfs}
\geometry{left=3cm,right=3cm,top=2.25cm,bottom=2.25cm} 


\renewcommand{\qedsymbol}{\hfill $\blacksquare$\par}
\renewcommand{\emptyset}{\varnothing}
\renewcommand{\labelitemii}{\textbullet}
\newcommand{\set}[1]{\left\{#1\right\}}
\newcommand{\staExp}[2][]{\mathbf{E}_{#1}\left[#2\right]}
\newcommand{\bigstaExp}[2][]{\mathbf{E}_{#1}\big[#2\big]}
\newenvironment{solution}{\begin{proof}[\indent\it Solution]}{\end{proof}}

\allowdisplaybreaks[4]

\setstretch{1.5}
\title{\textbf{Machine Learning Homework 01}}
\author{Qiu Yihang}
\date{Mar.27, 2022}

\begin{document}

\maketitle

\begin{proof}
We use $\boldsymbol{I}_p$ to denote $p-$dimensional unit matrix, i.e. $\boldsymbol{I}_p=\mathtt{diag}(1,1,...,1)_{p\times p}.$

\hspace{1.3em}
Since $\boldsymbol{V}$ is a $p\times p$ orthogonal matrix, we know $\boldsymbol{V}^{T}\boldsymbol{V}=\boldsymbol{V}\boldsymbol{V}^T=\boldsymbol{I}_p.$

\hspace{1.3em}
Thus, we have

\vspace{-2.5em}
\begin{align*}
    \left(\boldsymbol{Z}^T\boldsymbol{Z}+\lambda \boldsymbol{I}_p\right)^{-1}\boldsymbol{Z}^T\boldsymbol{y} &= \left(\left(\boldsymbol{UD}\boldsymbol{V}^T\right)^T\left(\boldsymbol{UD}\boldsymbol{V}^T\right)+\lambda \boldsymbol{V}^T\boldsymbol{V}\right)^{-1}\boldsymbol{Z}^T\boldsymbol{y} \\
    &= \left(\boldsymbol{V}\boldsymbol{D}^T\boldsymbol{U}^T\boldsymbol{UD}\boldsymbol{V}^T+\lambda \boldsymbol{V}^T\boldsymbol{I}_p\boldsymbol{V}\right)^{-1}\boldsymbol{Z}^T\boldsymbol{y} \\
    &= \left(\boldsymbol{V}\boldsymbol{D}^T\boldsymbol{D}\boldsymbol{V}^T+\boldsymbol{V}^T\left(\lambda\boldsymbol{I}_p\right)\boldsymbol{V}\right)^{-1}\boldsymbol{Z}^T\boldsymbol{y} \\
    &= \left(\boldsymbol{V}\left(\boldsymbol{D}^2 +\lambda \boldsymbol{I}_p\right)\boldsymbol{V}^T\right)^{-1}\boldsymbol{Z}^T\boldsymbol{y} 
\end{align*}

\vspace{-1.2em} \hspace{1.3em}
Meanwhile,

\vspace{-3em}
\begin{align*}
    &\quad\ \left(\boldsymbol{V}\left(\boldsymbol{D}^2 +\lambda \boldsymbol{I}_p\right)\boldsymbol{V}^T\right)\left(\boldsymbol{V}\underset{j}{\mathtt{diag}}\left(\frac{1}{d_j^2+\lambda}\right)\boldsymbol{V}^T\right) \\
    &=\ \boldsymbol{V}\underset{j}{\mathtt{diag}}\left(d_j^2+\lambda\right)\left(\boldsymbol{V}^T\boldsymbol{V}\right)\underset{j}{\mathtt{diag}}\left(\frac{1}{d_j^2+\lambda}\right)\boldsymbol{V}^T \\
    &=\ \boldsymbol{V}\underset{j}{\mathtt{diag}}\left(d_j^2+\lambda\right)\underset{j}{\mathtt{diag}}\left(\frac{1}{d_j^2+\lambda}\right)\boldsymbol{V}^T\  =\  \boldsymbol{V}\boldsymbol{I}_p\boldsymbol{V}^T=\boldsymbol{V}\boldsymbol{V}^T=\boldsymbol{I}_p,
\end{align*}

\vspace{-1em} \hspace{1.3em}
Therefore, 

\vspace{-2.5em}
\begin{align*}
    \left(\boldsymbol{Z}^T\boldsymbol{Z}+\lambda \boldsymbol{I}_p\right)^{-1}\boldsymbol{Z}^T\boldsymbol{y} 
    &= \left(\boldsymbol{V}\left(\boldsymbol{D}^2 +\lambda \boldsymbol{I}_p\right)\boldsymbol{V}^T\right)^{-1}\boldsymbol{Z}^T\boldsymbol{y} \\
    &= \boldsymbol{V}\underset{j}{\mathtt{diag}}\left(\frac{1}{d_j^2+\lambda}\right)\boldsymbol{V}^T\boldsymbol{Z}^T\boldsymbol{y} \\
    &= \boldsymbol{V}\underset{j}{\mathtt{diag}}\left(\frac{1}{d_j^2+\lambda}\right)\boldsymbol{V}^T\left(\boldsymbol{UD}\boldsymbol{V}^T\right)^T\boldsymbol{y} \\
    &= \boldsymbol{V}\underset{j}{\mathtt{diag}}\left(\frac{1}{d_j^2+\lambda}\right)\boldsymbol{V}^T\boldsymbol{V}\boldsymbol{D}^T\boldsymbol{U}^T\boldsymbol{y} \\
    &= \boldsymbol{V}\underset{j}{\mathtt{diag}}\left(\frac{1}{d_j^2+\lambda}\right)\underset{i}{\mathtt{diag}}\left(d_i\right)\boldsymbol{U}^T\boldsymbol{y} \\
    &= \boldsymbol{V}\underset{j}{\mathtt{diag}}\left(\frac{d_j}{d_j^2+\lambda}\right)\boldsymbol{U}^T\boldsymbol{y} \\
\end{align*}

\vspace{-3em} \hspace{38em}
\textit{Qed.}
\end{proof}


\end{document}