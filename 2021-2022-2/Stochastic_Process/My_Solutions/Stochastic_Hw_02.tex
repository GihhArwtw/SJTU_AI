\documentclass{article}
\usepackage[utf8]{inputenc}
\usepackage{amsmath}
\usepackage{amsfonts}
\usepackage{setspace}
\usepackage{amsthm}
\usepackage{amssymb}
\usepackage{geometry}
\usepackage{verbatim}
\usepackage{mathrsfs}
\geometry{left=3cm,right=3cm,top=2.25cm,bottom=2.25cm} 


\renewcommand{\qedsymbol}{\hfill $\blacksquare$\par}
\renewcommand{\emptyset}{\varnothing}
\renewcommand{\labelitemii}{\textbullet}
\newcommand{\set}[1]{\left\{#1\right\}}
\newcommand{\staExp}[2][]{\mathbf{E}_{#1}\left[#2\right]}
\newcommand{\bigstaExp}[2][]{\mathbf{E}_{#1}\big[#2\big]}
\newenvironment{solution}{\begin{proof}[\indent\it Solution]}{\end{proof}}

\allowdisplaybreaks[4]

\setstretch{1.5}
\title{\textbf{Stochastic Process Homework 02}}
\author{Qiu Yihang}
\date{Mar.07 - Mar.22, 2022}

\begin{document}

\maketitle

\setcounter{section}{-1}
\section{Reference}
\hspace{2em}
This homework is completed with the help of discussions with Sun Yilin and Ji Yikun. The completion of \textbf{2.3} is based on the notes of AI2613 last year.

\section{Optimal Coupling}
\vspace{1em}
\begin{proof}
Since

\vspace{-2em}
\begin{align}
    D_{\mathrm{TV}}\left(\mu,\nu\right)=\frac{1}{2}\sum_{i\in\Omega}\left|\mu(i)-\nu(i)\right|=\sum_{i\in\Omega,\nu(i)\geq\mu(i)}\left(\nu(i)-\mu(i)\right),\label{D_TV}
\end{align}

\vspace{-0.5em} \hspace{1.3em}
it is natural to maximize $\omega(i,i)$, i.e. set $\omega(i,i)=\min\left\{\mu(i),\nu(i)\right\}$.

\hspace{1.3em}
Considering $\mu(i)-\min\left\{\mu(i),\nu(i)\right\}=\max\left\{0,\mu(i)-\nu(i)\right\}, \nu(j)-\min\left\{\mu(j),\nu(j)\right\}=\max\{0,$ $\nu(j)-\mu(j)\}$, we construct a feasible coupling $\omega$ as follows.

$$\omega(i,j)=\left\{
\begin{array}{ll}
    \min\left\{\mu(i),\nu(i)\right\}, & i=j \\
    C\max\left\{0,\mu(i)-\nu(i)\right\}\max\left\{0,\nu(j)-\mu(j)\right\}, & i\neq j
\end{array}\right.$$

\vspace{3em} \hspace{1.3em}
Now we solve $C$. We know

\vspace{-2.5em}
\begin{align*}
    \mu(i) &= \sum_{j\in\Omega}\omega(i,j) \\
    &= \sum_{j\neq i\in\Omega}C\max\left\{0,\mu(i)-\nu(i)\right\}\max\left\{0,\nu(j)-\mu(j)\right\} + \min\left\{\mu(i),\nu(i)\right\} \\
    &= \min\left\{\mu(i),\nu(i)\right\} + C\max\left\{0,\mu(i)-\nu(i)\right\}\sum_{j\neq i\in\Omega}\max\left\{0,\nu(j)-\mu(j)\right\}
\end{align*}

\vspace{-1em} \hspace{3.9em}
\textbf{CASE 01.} When $\min\left(\mu(i),\nu(i)\right)=\mu(i)$, obvious the equation above holds.

\hspace{3.9em}
\textbf{CASE 02.} When $\min\left(\mu(i),\nu(i)\right)=\nu(i)$, 

\hspace{3.9em}
by \eqref{D_TV}, we have

\vspace{-2.5em}
\begin{align*}
    \mu(i) &= \nu(i) + C\left(\mu(i)-\nu(i)\right)\sum_{j\neq i\in\Omega, \nu(j)\geq\mu(j)}\left(\nu(j)-\mu(j)\right) \\
    &= \nu(i) + C\left(\mu(i)-\nu(i)\right)D_{\mathrm{TV}}\left\{\mu,\nu\right\}
\end{align*}

\vspace{-1em}
\hspace{3.9em}
Thus,

\vspace{-1em}
$$C=\frac{1}{D_{\mathrm{TV}}\left(\mu,\nu\right)}$$

\vspace{-1em} \hspace{3.9em}
i.e.

\vspace{-2em}
\begin{align*}
    \omega(i,j)=\left\{
    \begin{array}{ll}
    \min\left\{\mu(i),\nu(i)\right\}, & i=j \\
    \frac{1}{D_{\mathrm{TV}}\left(\mu,\nu\right)}\max\left\{0,\mu(i)-\nu(i)\right\}\max\left\{0,\nu(j)-\mu(j)\right\}, & i\neq j
    \end{array}\right.
\end{align*}

\vspace{-0.5em} \hspace{3.9em}
By the symmetry of $\mu$ and $\nu$, $i$ and $j$ in the equation, we know 

\vspace{-1.2em}
$$\nu(j) = \sum_{i\in\Omega}\omega(i,j)$$

\vspace{-0.5em} \hspace{3.9em}
also holds. Thus, $\omega$ is a valid coupling.

\vspace{3em} \hspace{1.3em}
Now we prove $\mathbf{Pr}_{(X,Y)\sim\omega}\left[X\neq Y\right]=D_{\mathrm{TV}}\left(\mu,\nu\right)$.

\vspace{-2.5em}
\begin{align*}
    \mathbf{Pr}_{(X,Y)\sim\omega}\left[X=Y\right] &= \sum_{i\in\Omega}\omega_(i,i) \\
    &= \sum_{i\in\Omega}\min\left\{\mu(i),\nu(i)\right\} \\
    &= \sum_{i\in\Omega, \mu(i)\geq \nu(i)}\nu(i) + \sum_{i\in\Omega, \mu(i)<\nu(i)}\mu(i) \\
    &= \sum_{i\in\Omega} \nu(i) + \sum_{i\in\Omega, \nu(i)\geq\mu(i)}(\mu(i)-\nu(i)) \\
    &= 1 - \sum_{i\in\Omega, \nu(i)\geq\mu(i)}(\nu(i)-\mu(i)) \\
    &= 1 - D_{\mathrm{TV}}\left(\mu,\nu\right)\qquad\qquad\qquad\qquad(\mathrm{by\ }\eqref{D_TV})
\end{align*}

\vspace{-1.2em} \hspace{3.9em}
i.e.,

\vspace{-3em}
\begin{align*}
    \mathbf{Pr}_{(X,Y)\sim\omega}\left[X\neq Y\right] = 1 - \mathbf{Pr}_{(X,Y)\sim\omega}\left[X=Y\right] = D_{\mathrm{TV}}\left(\mu,\nu\right).
\end{align*}

\vspace{2em} \hspace{1.3em}
Thus, there exists a coupling

\vspace{-1.8em}
\begin{align*}
    \omega(i,j)=\left\{
    \begin{array}{ll}
    \min\left\{\mu(i),\nu(i)\right\}, & i=j \\
    \frac{1}{D_{\mathrm{TV}}\left(\mu,\nu\right)}\max\left\{0,\mu(i)-\nu(i)\right\}\max\left\{0,\nu(j)-\mu(j)\right\}, & i\neq j
    \end{array}\right.
\end{align*}

\hspace{1.3em}
s.t. $\mathbf{Pr}_{(X,Y)\sim\omega}\left[X\neq Y\right]=D_{\mathrm{TV}}\left(\mu,\nu\right)$.
\end{proof}

\newpage


\section{Stochastic Dominance}
\vspace{1em}

\subsection{Binomial Distribution Case}
\vspace{1em}

\begin{proof}
First we prove the sufficiency.

\hspace{1.3em}
Assume $p\geq q$. Let $X\sim\mathtt{Binom}(n,p), Y\sim\mathtt{Binom}(n,q).$ 

\hspace{1.3em}
For any $a\in\Omega$,

\hspace{1.3em}
\textbf{CASE 01.} When $a\geq n+1$, $\mathbf{Pr}\left[X\geq a\right]=\mathbf{Pr}\left[Y\geq a\right]=0.$

\hspace{1.3em}
\textbf{CASE 02.} When $a\le -1$, $\mathbf{Pr}\left[X\geq a\right]=\mathbf{Pr}\left[Y\geq a\right]=1.$

\hspace{1.3em}
\textbf{CASE 03.} When $1\le a\le n$, we have

\vspace{-2.5em}
\begin{align*}
   \mathbf{Pr}\left[X\geq a\right] &= \sum_{i=a}^{n}\mathbf{Pr}\left[X=i\right] = \sum_{i=a}^n\binom{n}{i}p^i(1-p)^{n-i}
   \\
   \mathbf{Pr}\left[Y\geq a\right] &= \sum_{i=a}^{n}\mathbf{Pr}\left[Y=i\right] = \sum_{i=a}^n\binom{n}{i}q^i(1-q)^{n-i}
\end{align*}

\vspace{-1em} \hspace{1.3em}
Consider $f(x)=\sum_{i=a}^{n}\binom{n}{i}x^i(1-x)^{n-i}.$ When $x\in[0,1],$

\vspace{-2.25em}
\begin{align*}
    \frac{\mathrm{d}f}{\mathrm{d}x} &= nx^{n-1}+\sum_{i=a}^{n-1} \binom{n}{i}\left(ix^{i-1}(1-x)^{n-i}-(n-i)x^i(1-x)^{n-i-1}\right) \\
    &= nx^{n-1}+\sum_{i=a}^{n-1}\binom{n}{i}\left(i-nx)x^{i-1}(1-x)^{n-i-1}\right) \geq 0
\end{align*}

\vspace{-1em} \hspace{6em}
i.e. $f(x)$ is monotonously increasing on $[0,1].$ 

\vspace{0.3em} \hspace{1.3em}
Since $p,q\in[0,1],\ p\geq q$, we know $f(p)\geq f(q)$, i.e. 

\vspace{-1.5em}
$$\mathbf{Pr}\left[X\geq a\right]\geq\mathbf{Pr}\left[Y\geq a\right].$$

\vspace{-0.75em} \hspace{1.3em}
Thus, for any $a\in\Omega,\ \mathbf{Pr}\left[X\geq a\right]\geq\mathbf{Pr}\left[Y\geq a\right]$, i.e. $\mathtt{Binom}(n,p)\succeq\mathtt{Binom}(n,q).$

\vspace{5em} \hspace{1.3em}
Now we prove the necessity.

\hspace{1.3em}
Assume $\mathtt{Binom}(n,p)\succeq\mathtt{Binom}(n,q)$. Let $X\sim\mathtt{Binom}(n,p), Y\sim\mathtt{Binom}(n,q).$ 

\hspace{1.3em}
Then we have

\vspace{-2.5em}
\begin{align*}
    \mathbf{Pr}\left[X\geq n\right]\geq\mathbf{Pr}\left[Y\geq n\right] &\Longleftrightarrow \mathbf{Pr}\left[X= n\right]\geq\mathbf{Pr}\left[Y= n\right] \\
    &\Longleftrightarrow p^n \geq q^n \\
    &\Longleftrightarrow p \geq q.
\end{align*}

\vspace{3em}
\hspace{1.3em}
In conclusion, for any $p,q\in[0,1], \mathtt{Binom}(n,p)\succeq\mathtt{Binom}(n,q)$ \textbf{iff.} $p\geq q.$
\end{proof}


\subsection{Monotone Coupling}
\vspace{1em}
\begin{proof}
    First we prove the sufficiency.
    
    \hspace{1.3em}
    Assume exists a monotone coupling $\omega$ of $\mu$ and $\nu$. Then we know 
    
    \vspace{-1.5em}
    $$1 = \mathbf{Pr}_{(X,Y)\sim\omega}\left[X\geq Y\right] = \sum_{i\in\Omega} \sum_{j\le i}\omega(i,j) = 1 = \sum_{i\in\Omega}\sum_{j\in\Omega}\omega(i,j)$$
    
    \vspace{-1em} \hspace{1.3em}
    Thus, when $i<j,\ \omega(i,j)=0.$
    
    \vspace{-3em}
    \begin{align*}
        \forall a\in \Omega,\quad \mathbf{Pr}_{X\sim\mu}\left[X\geq a\right] &= \mathbf{Pr}_{(X,Y)\sim\omega}\left[X\geq a\right] = \sum_{i\in\Omega, i\geq a\ }\mathbf{Pr}_{(X,Y)\sim\omega}\left[X=i\right] \quad \\
        &= \sum_{i\in\Omega, i\geq a\ }\sum_{j\in\Omega}\omega(i,j) 
        \ = \sum_{j\in\Omega\ } \sum_{i\in\Omega, i\geq a} \omega(i,j) \\
        &\geq \sum_{j\in\Omega, j\geq a\ } \sum_{i\in\Omega, i\geq j} \omega(i,j) \ \ = \sum_{j\in\Omega, j\geq a\ } \sum_{i\in\Omega, i\geq j} \omega(i,j) + 0\\
        & = \sum_{j\in\Omega, j\geq a\ } \sum_{i\in\Omega, i\geq j} \omega(i,j)\ + \sum_{j\in\Omega, j\geq a\ } \sum_{i\in\Omega, i<j} \omega(i,j) \\
        &= \sum_{j\in\Omega, j\geq a\ } \sum_{i\in\Omega} \omega(i,j)\  = \sum_{j\in\Omega, j\geq a} \mathbf{Pr}_{(X,Y)\sim\omega}\left[Y=j\right] \\
        &= \mathbf{Pr}_{(X,Y)\sim\omega}\left[Y\geq a\right] \\
        &= \mathbf{Pr}_{Y\sim\nu}\left[Y\geq a\right].
    \end{align*}

    \vspace{-1em} \hspace{1.3em}
    i.e. \qquad $\mu\succeq\nu.$
    
    
    \vspace{3em} \hspace{1.3em}
    Now we prove the necessity, i.e. to construct a coupling $\omega$ s.t. $\mathbf{Pr}_{(X,Y)\sim\omega}\left[X\geq Y\right]=1.$
    
    \hspace{1.3em}
    Assume $\mu\succeq\nu$. We construct $\omega$ as follows. 
    
    \hspace{1.3em}
    First we set $\omega(i,j)=0$ for $i<j.$
    
    \hspace{1.3em}
    We have $\mu(n)=\mathbf{Pr}_{X\sim\mu}\left[X\geq n\right]\geq\mathbf{Pr}_{Y\sim\nu}\left[Y\geq n\right]=\nu(n).$ Set $\omega(n,n)=\nu(n).$
    
    \hspace{1.3em}
    For the remaining part, we determine $\omega(i,j)$ in the following order
    
    \vspace{-3em}
    \begin{align*}
        &\omega(n,n-1),\omega(n-1,n-1),\\
        &\omega(n,n-2),\omega(n-1,n-2),\omega(n-2,n-2),
        \\
        &...,\\
        &\omega(n,1),\omega(n-1,1),...,\omega(1,1)
    \end{align*}
    
    \vspace{-1em} \hspace{1.3em}
    by the following method.
    
    \vspace{-2.5em}
    \begin{align*}
        \omega(i,j) = \min\left\{\mu(i)-\sum_{k=j+1}^n\omega(i,k),\quad \nu(j)-\sum_{k=i+1}^n\omega(k,j)\right\}
    \end{align*}
    
    \vspace{-0.5em} \hspace{1.3em}
    We prove $\omega$ is a valid coupling as follows. Obvious $\mu(i)=\sum_{j\in\Omega}\omega(i,j),\nu(j)=\sum_{j\in\Omega}\omega(i,j),$ which is maintained and guaranteed by how we determined $\omega(i,j)$.
    
    \hspace{1.3em}
    Meanwhile, $\omega(i,j)\geq0,$ (since $\sum_{k=j+1}^n\omega(i,k)\le\mu(i),\sum_{k=i+1}^n\omega(k,j)\le\nu(j)$).
    
    \hspace{1.3em}
    Thus, $\omega$ is a valid coupling s.t. $\mathbf{Pr}_{(X,Y)\sim\omega}\left[X\geq Y\right]=1.$
    
    \vspace{2em} \hspace{1.3em}
    In conclusion, $\mu\succeq\nu$ \textbf{iff.} exists a coupling $\omega$ s.t. $\mathbf{Pr}_{(X,Y)\sim\omega}\left[X\geq Y\right]=1.$
\end{proof}

\vspace{9em}
\subsection{Erdős–Rényi Model Random Graph}
\vspace{1em}

\begin{itemize}
    \item \textit{The completion of this problem is based on the notes of}\textbf{ AI2613 }\textit{last year.}
\end{itemize}

\vspace{2em}
\begin{proof}
We can generate $G\sim\mathcal{G}(n,p)$ and $H\sim\mathcal{G}(n,q)$ simultaneously, where $p,q\in[0,1],\ p\geq q.$

\hspace{1.3em}
For each pair of vertices $(u,v)$, we independently pick $r_{\left\{u,v\right\}}\sim\mathtt{Uniform}(0,1)$. We determine whether graph $G$ and $H$ has edge $\left\{u,v\right\}$ as follows.

\vspace{-2em}
\begin{align*}
    \left\{\begin{array}{ll}
        \mathrm{both\ }G\mathrm{\ and\ }H\mathrm{\ have\ edge}\left\{u,v\right\}, & r_{\left\{u,v\right\}}\in[0,q] \\
        \mathrm{only\ }G\mathrm{\ has\ edge}\left\{u,v\right\}, & r_{\left\{u,v\right\}}\in(q,p] \\
        \mathrm{neither\ has\ edge}\left\{u,v\right\}, & r_{\left\{u,v\right\}}\in(p,1]
    \end{array}\right.
\end{align*}

\vspace{-0.5em} \hspace{1.3em}
Obvious $H$ is always a subgraph of $G$. If $H$ is connected, $G$ is for sure connected.

\hspace{1.3em}
Thus, $\mathbf{Pr}\left[G\mathrm{\ is\ connected}\right]\geq\mathbf{Pr}\left[H\mathrm{\ is\ connected}\right]$.

\vspace{1em} \hspace{1.3em}
Therefore, for any $p,q\in[0,1]$ s.t. $p\geq q$, 

\vspace{-1em}
$$\mathbf{Pr}_{G\sim\mathcal{G}(n,p)}\left[G\mathrm{\ is\ connected}\right]\geq\mathbf{Pr}_{H\sim\mathcal{G}(n,q)}\left[H\mathrm{\ is\ connected}\right].$$

\hspace{38em}
\textit{Qed.}
\end{proof}


\newpage

\section{Total Variation Distance is Non-Increasing}
\vspace{1em}

\begin{proof}
    By \textbf{Coupling Lemma}, we know
    
    \vspace{-1em}
    $$\Delta(t)\le\mathbf{Pr}_{(X,Y)\sim\omega_t}\left[X\neq Y\right],$$
    
    \vspace{-0.3em}\hspace{5.3em}
    where $\omega_t$ is a coupling of $\mu_t$ and $\pi$.
    
    \vspace{0.7em} \hspace{1.3em}
    We construct a coupling $\omega_{t}$ as follows, where $(X_t,Y_t)\sim\omega_t,(X_{t+1},Y_{t+1})\sim\omega_{t+1}$.
    
    \vspace{-2.2em}
    \begin{align*}
        \left\{\begin{array}{ll}
            X_{t+1}=X_t=Y_t=Y_{t+1}, & \mathrm{if\ }X_t=Y_t \\
            X_{t+1}\sim\mu_t,\ Y_{t+1}\sim\pi, & \mathrm{if\ }X_t\neq Y_t
        \end{array}\right.
    \end{align*}
    
    \hspace{1.3em}
    Then we have
    
    \vspace{-3em}
    \begin{align*}
        \Delta(t+1) &\le \mathbf{Pr}_{(X,Y)\sim\omega_{t+1}}\left[X\neq Y\right] \\
        &= \mathbf{Pr}\left[X_{t+1}\neq Y_{t+1}\right] \\
        &\le \mathbf{Pr}\left[X_{t}\neq Y_{t}\right] \\
        &= \Delta(t).
    \end{align*}
    
    \hspace{38em}
    \textit{Qed.}
\end{proof}





\end{document}