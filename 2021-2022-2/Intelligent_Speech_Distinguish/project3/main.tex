\documentclass[a4paper]{article}

\usepackage{ReportTemplate}

\usepackage{setspace}
\usepackage{amsmath}
\usepackage[hidelinks]{hyperref}

\usepackage{multirow}

\renewcommand{\tt}[1]{\mathtt{#1}}
\newcommand{\bd}[1]{\boldsymbol{#1}}

\title{Project 3:LVCSR系统搭建}
\name{邱一航}
\studentid{520030910155}

\begin{document}

\maketitle

\section{Baseline (70分)}

\setcounter{subsection}{-1}
\subsection{安装环境及脚本运行说明}

笔者在Windows提供的Linux子系统即WSL(Windows Subsystem for Linux)环境下进行实验。

所安装的WSL版本为2.0,笔者在该子系统中安装Ubuntu操作系统,Ubuntu版本为Ubuntu 20.04。WSL环境下已预先安装了CUDA 10.1。

该部分只要求训练得到SAT+GMM-HMM。因此原脚本中涉及后续DNN训练的部分应当被注释,即注释以下两行(Line 142, 145)。

\begin{lstlisting}[language=bash,firstnumber=142]
local/nnet3/run_tdnn.sh
\end{lstlisting}
\vspace{-1em}
\begin{lstlisting}[language=bash,firstnumber=145]
local/chain/run_tdnn.sh
\end{lstlisting}

\vspace{-0.3em}
\subsection{数据处理及特征提取(10分)}

官方recipe的数据处理和特征提取步骤如下:

\vspace{1em}
\textbf{1. 词典准备。} 

调用$\tt{local}/\tt{aishell}\_\tt{prepare\_dict.sh}$,根据提供的词库$(\tt{data/resource\_aishell}$ $\tt{/lexicon.txt})$准备词典。

\vspace{1em}
\textbf{2. 数据准备。} 

调用$\tt{local/aishell\_data\_prep.sh}$,根据提供的文本标注文件$\tt{data/data\_aishell}$ $\tt{/transcript/aishell\_transcript\_v0.8.txt}$在train, test, dev数据集上准备文本标注及speaker与语音编号对应关系。

\vspace{1em}
\textbf{3. 词典生成。} 

调用$\tt{utils/prepare\_lang.sh}$,根据准备好的词典文件,以标准格式构建词典的FST脚本。

\vspace{1em}
\textbf{4. 语言模型训练}。

调用$\tt{local/aishell\_train\_lms.sh}$,进行语言模型训练。

\vspace{1em}
\textbf{5. 特征提取。}

用$\tt{steps/make\_mfcc\_pitch.sh},\tt{steps/compute}$ $\tt{\_cmvn\_stats.sh}, \tt{utils/fix\_data\_dir.sh}$提取MFCC(Mel频率倒谱系数)和Pitch(音高)特征。其中第二个脚本用于归一化,第三个脚本检查$\tt{wav.scp}$文件格式的正确性。

\vspace{1.5em}
该过程曾出现以下错误信息:

\begin{lstlisting}[language=bash]
local/aishell_train_lms.sh: train_lm.sh is not found. That might mean it's not installed.
\end{lstlisting}

\textbf{解决}:根据提示使用$\tt{tools/extras/install}$ $\tt{\_kaldi\_lm.sh}$安装$\tt{train\_lm.sh}$即可。

\subsection{模型训练(50分)}

官方recipe中,SAT+GMM-HMM模型训练主要分为两个阶段:

\subsubsection{单音素训练}

利用MFCC和pitch特征进行单音素的GMM-HMM模型训练,利用最大似然法估计参数。使用EM(期望最大化)算法,默认迭代40次(可以在$\tt{steps/train\_mono.sh}$脚本中修改迭代次数),每次迭代增加一定的高斯数并重新进行参数估计。

特别地,脚本通过${realign\_iters}$变量指定了一些需要执行对齐的迭代次数。

训练完成后,生成解码图并利用该模型在数据集上解码,计算WER和CER。之后使用Verrtibi算法对解码后的特征进行对齐,以便之后的训练使用。下面简称该过程为“解码、评估、对齐”。

\subsubsection{Triphone(三音素)训练1-2:train\_deltas,即一阶与二阶差分} 

第一阶段和第二阶段均调用$\tt{steps/train\_}$ $\tt{deltas.sh}$,构建以GMM为节点的决策树。训练过程中也使用了上一次训练后对齐的特征。

每一阶段的训练完成后,都进行了一次“解码、评估、对齐”。

\vspace{1em}
这两个阶段训练调用的脚本$\tt{steps/train\_}$ $\tt{deltas.sh}$的主要流程如下:

\begin{itemize}
    \item 首先对特征进行处理(CMVN,倒谱均值方差归一化方法)和差分。即对特征的倒谱值进行处理
    
    \vspace{-2em}
    \begin{align*}
        \hat{x}_i = \frac{x_i-\tt{mean}(x)}{\sqrt{\tt{var}(x)}}
    \end{align*}
    
    后,再进行差分操作。
    
    这一处理能提高系统的泛化能力,提升了模型的鲁棒性。
    
    \item \textbf{stage -3:} 利用前一次训练得到的对齐结果,统计收集所有决策树的参数,结果在treeacc文件中。
    \item \textbf{stage -2:} 使用k-means对三音素进行聚类,根据聚类结果和上一步统计完毕的决策树,利用前一次训练得到的模型(tri1:单因素模型,tri2:tri1得到的三音素决策树),构建基于三音素的决策树并初始化。决策树的每个节点都是一个GMM模型。
    \item \textbf{stage -1:} 
    将上一步的对齐(tri1:单因素表示的对齐,tri2:tri1得到的三音素决策树节点表示的对齐)转换为用上一步构建的三音素决策树的节点表示的对齐。
    \item \textbf{stage 0:}
    构建三音素决策树的训练图。
    \item 此后类似\textbf{1.2.1}中单音素GMM的训练,使用EM算法,通过最大化似然来估计每个节点处GMM的具体参数。
    
    同样,也通过${realign\_iters}$变量指定了需要对齐的迭代次数。
    
\end{itemize}

第一阶段和第二阶段分别针对MFCC特征经CMVN后一阶和二阶差分特征进行建模,完成了两个模型(三音素决策树)的建构和训练。

\vspace{2em}

\subsubsection{Triphone(三音素)训练3:LDA+MLLT}

第三阶段生成并训练的模型为:对splice(帧链接)处理后的MFCC和pitch特征进行LDA降维,降维后再做MLLT变换,通过以GMM-HMM为节点的决策树得到结果。

\vspace{1em}
该阶段调用的脚本$\tt{steps/train\_lda\_mllt.}$ $\tt{sh}$的主要流程如下:

\begin{itemize}
    \item 对MFCC和pitch特征进行CMVN处理,然后使用splice进行对处理后的特征进行帧链接。拼接后的特征相当于三音素的“特征”。
    \item \textbf{stage -5:} 使用LDA(线性判别分析)对拼接后的特征进行降维和去相关(默认降维至40维)。LDA的具体流程在机器学习课程\cite{ML}中已有详细讲解,此处只给出方法的核心。
    
    1) 构建类内散射矩阵:
    
    \vspace{-0.5em}
    $$S_w=\sum_{c}\sum_{\bf{x}\in H_c}(\bf{x}-\bd{\mu}_c)(\bf{x}-\bd{\mu}_c)^T$$
    
    其中$c$为类别,$H_c$包括了第$c$类的全部数据点,$\bd{\mu}_c$为第$c$类全部数据点均值(中心)。
    
    2) 通过计算全局散度矩阵
    
    \vspace{-2em}
    $$S_T=\sum_{\bf{x}}(\bf{x}-\bd{\mu})(\bf{x}-\bd{\mu})^T=S_b+S_w$$
    
    \vspace{-1em}
    计算类间散度矩阵$S_b$。其中$\bd{\mu}$为所有数据点的均值。
    
    3) LDA的目标是最大化类间距离同时最小化类内距离,即
    
    \vspace{-1.5em}
    $$\max J=\frac{\bd{w}^TS_w\bd{w}}{\bd{w}^TS_b\bd{w}}\ \Longrightarrow\ S_w^{-1}S_b\bd{w}=J\bd{w}$$
    
    其中$\bd{w}$为降维后的超平面的一个基,即将数据点$\bd{x}$投影到$\bd{w}^T\bf{x}$。该目标可转化为上式右侧的特征值分解问题。多个$\bd{w}$可以组成一个降维变换矩阵$\bf{W}$,很显然各个$\bd{w}$都是特征向量,彼此不相关。
    
    \item \textbf{stage -4 - stage 0} 
    对三音素聚类,统计决策树参数,构建以GMM为节点的决策树并初始化。
    
    \item 用LDA降维和去相关后,类似$\tt{train\_}$ $\tt{deltas.sh}$,通过EM算法多次迭代最大似然估计得到“最优”的MLLT(最大似然线性变换)矩阵\cite{LDA}和以GMM-HMM为节点的三音素决策树模型。这一训练过程类似\textbf{1.2.2}中的训练,不再赘述。
    
    由于特征维度有40,GMM所需要的协方差矩阵参数量太大,因此使用MLLT矩阵,即通过相对较小的对角阵$\bd{\Sigma}_{diag}$和semi-tied covariance matrices $\bd{H}$来近似估计GMM中的协方差矩阵。数学表达式如下:
    
    \vspace{-1em}
    $$\bd{\Sigma}=\bd{H}\bd{\Sigma}_{diag}\bd{H}^T$$
    
    为了方便EM算法的使用(每一次迭代都会使用上一次的迭代结果),同时对特征和GMM做这一处理,即
    
    \vspace{-2em}
    \begin{align*}
        \left\{
        \begin{array}{l}
            \hat{\bd{\mu}} = \bd{H}\bd{\mu}  \\
            \hat{\bd{\Sigma}}=\bd{H}\bd{\Sigma}\bd{H}^T
        \end{array}
        \right.
    \end{align*}
    
    因此,MLLT可视为特征空间的变换矩阵。
\end{itemize}

\vspace{1em}
训练完成后,需要为后续SAT训练准备特征。这一步的目的是将原本数据集中的特征(是与说话人“无关”的特征)根据特定的说话人进行变换(对特征的均值进行一定变换处理,即

\vspace{-1em}
$$\bd{\mu}_{SA}=\bd{A}\bd{\mu}+\bd{b}$$

($\bd{\mu}$是原本的均值,$\bd{b}$是偏移量,$\bd{A}$是特征空间中的变换即一个矩阵。)

这一变换生成了说话人自适应的音频。对变换$\bd{A}$的求解使用了fMLLR,该方法能在训练得到的模型上能取得最大似然。\cite{LDA}

$\tt{steps/align\_fmllr.sh}$使用EM算法迭代求解fMLLR,默认迭代次数是两次。



\vspace{1em}
\subsubsection{Triphone(三音素)训练4-5:SAT}

tri4a和tri5a都训练了SAT(说话人自适应训练)。该部分主要是根据说话人声音的特征(如性别,平均音高等)对音频特征进行一定处理,根据不同的“说话人”进行不同的处理。

某种意义上这一操作可以被视为针对“说话人”的归一化,将原本不同的声线根据“说话人”自适应得到的变换,全部转化为“标准声线”下的音频进行后续识别处理。tri4a和tri5a所训练就是说话人自适应,即不同的声线该如何转化为“标准声线”。

tri4a训练的SAT模型相对较小,tri5a训练的SAT模型参数量更大,高斯数也更多。

其训练流程类似于前两阶段的训练,也是训练一个节点为GMM的三音素决策树。唯一的区别是输入的特征都经过了fMLLR处理。因此这里不再赘述。

两次训练完成后都需要经过fMLLR变换,以便进行下一次的SAT训练。

%在本节中,请运行并介绍官方recipe中从最初没有模型,直至训练出一个SAT + GMM-HMM模型的流程(可选:你遇到了什么问题?描述问题原因并介绍你的解决方法)。你可能需要查阅资料(尤其是SAT部分)以及与同学讨论才能理解这一流程,但请以自己的理解进行描述。\textbf{写作完成后,请删除本段。}

\subsection{实验结果(10分)}

不同阶段训练得到的模型在test测试集和dev开发集上的解码结果保存在$\tt{exp/}[1]\tt{/decode}$ $\_[2]\tt{/log/decode.}[3]\tt{.log}$文件中。其中,[1]是训练阶段的名称(mono, tri1, tri2, tri3a, tri4a, tri5a),[2]是数据集名称(test, dev),[3]是数值(1-6).

不同阶段训练得到的系统对应的评分结果相对目录如表\ref{tab1}:(均为test数据集上的结果)

\begin{table}[th]
  %\hspace{-5em}
  \begin{tabular}{ c l }
    \toprule
    \textbf{阶段} & \textbf{相对路径}  \\
    \midrule
    mono & $\tt{exp/mono/decode\_test/scoring\_kaldi}$  \\
    tri1 & $\tt{exp/tri1/decode\_test/scoring\_kaldi}$  \\
    tri2 & $\tt{exp/tri2/decode\_test/scoring\_kaldi}$  \\
    tri3a & $\tt{exp/tri3a/decode\_test/scoring\_kaldi}$  \\
    tri4a & $\tt{exp/tri4a/decode\_test/scoring\_kaldi}$  \\
    tri5a & $\tt{exp/tri5a/decode\_test/scoring\_kaldi}$  \\
    \midrule
    tri4a-si & $\tt{exp/tri4a/decode\_test.si/scoring\_kaldi}$  \\
    tri5a-si & $\tt{exp/tri5a/decode\_test.si/scoring\_kaldi}$  \\
    \bottomrule
  \end{tabular}
  \vspace{0.5em}
  \centering \caption{Baseline不同阶段评分结果相对路径}
  \label{tab1}
\end{table}

\vspace{-1em}
评分结果为上述路径下的$\tt{best\_wer}$和$\tt{best}$ $\_\tt{cer}$ 文件,分别为WER和CER。

\vspace{1em}
不同阶段训练结果对应的性能对比如表\ref{tab2}:

(以下评分均为test测试集上的评分结果)

\begin{table}[th]
  \centering
  \begin{tabular}{ c | c c }
    \toprule
    \textbf{阶段} & \textbf{WER(\%)} & \textbf{CER(\%)} \\
    \midrule
    mono & 57.98 & 45.30 \\
    tri1 & 43.64 & 28.50 \\
    tri2 & 43.77 & 28.56 \\
    tri3a & 41.23 & 25.81 \\
    \textbf{tri4a} & \textbf{36.33} & \textbf{20.72} \\
    tri5a & 38.01 & 22.38 \\
    \midrule
    tri4a-si & 42.80 & 27.56 \\
    tri5a-si & 44.20 & 28.81 \\
    \bottomrule
  \end{tabular}
  \vspace{0.5em}
  \centering \caption{Baseline不同阶段训练结果对比}
  \label{tab2}
\end{table}


显然,tri4a阶段训练后的系统性能最佳。

tri5a的评分结果不如tri4a的可能原因是经过tri5a训练后SAT模型过拟合,导致泛化能力不如tri4a训练后的系统,性能下降。

特别地,tri4a-si和tri5a-si可以看成消融实验,证明对不同说话人进行特定的fMMLR变换对降低WER和CER、提升系统性能有重要作用。




%WER  [ 47461 / 104765, 1063 ins, 2546 del, 43852 sub ] exp/mono/decode_test/cer__0.0
%WER 28.50 [ 29853 / 104765, 1116 ins, 1479 del, 27258 sub ] exp/tri1/decode_test/cer_14_0.5
%WER 28.56 [ 29921 / 104765, 1331 ins, 1327 del, 27263 sub ] exp/tri2/decode_test/cer_13_0.5
%WER 25.81 [ 27043 / 104765, 983 ins, 1349 del, 24711 sub ] exp/tri3a/decode_test/cer_13_1.0
%WER 20.72 [ 21707 / 104765, 812 ins, 923 del, 19972 sub ] exp/tri4a/decode_test/cer_14_0.5
%WER 22.38 [ 23444 / 104765, 989 ins, 1154 del, 21301 sub ] exp/tri5a/decode_test/cer_16_1.0


\section{语料数量对模型性能的影响(15分)}

%在本章中,你应该尝试将模型解码时使用的语言模型更换为由更大的原AIShell-1训练集语料(即压缩包中给出的\texttt{train\_large.txt})训练的大语言模型,并对比语言模型的大小对系统性能造成的影响。通过本章,你应该学会如何在Kaldi recipe中使用外部语料。\textbf{写作完成后,请删除本段。}

\subsection{修改方法}

为使用更大的训练集语料(原本的AIShell-1训练集语料),需对语言模型训练脚本做一定修改。修改$\tt{local/aishell\_train\_lms.sh}$的第7行

\begin{lstlisting}[language=bash,firstnumber=7]
text=data/local/train/text
\end{lstlisting}

为

\begin{lstlisting}[language=bash,firstnumber=7]
text=data/data_aishell/transcript/train_large.txt
\end{lstlisting}

之前的词典准备、数据准备、词典构建的结果依然可以继续沿用(没有发生变化),不必重新执行$\tt{local/aishell\_prepare\_dict.sh}$,$\tt{local/}$ $\tt{aishell\_data\_prep.sh}$, $\tt{utils/prepare\_lang.}$  $\tt{sh}$。数据准备和特征提取部分只需要重新执行语言模型的训练即可。

更换语料后,只需要重新执行\textbf{语言模型训练}步骤及模型训练部分的代码即可。

\subsection{性能对比}

换用大语料前后,SAT+GMM-HMM系统的性能对比如表\ref{tab3}。

(仅给出test测试集上的最优结果)

\begin{table}[th]
  \centering
  \begin{tabular}{ c | c c | c c}
    \toprule
    \multirow{2}*{\textbf{阶段}} & \multicolumn{2}{c|}{\textbf{baseline(小语料)}} & \multicolumn{2}{c}{\textbf{大语料}} \\
    & \textbf{WER\%} & \textbf{CER\%} & \textbf{WER\%} & \textbf{CER\%}\\
    \midrule
    mono & 57.98 & 45.30 & 45.37 & 36.01 \\
    tri1 & 43.64 & 28.50 & 32.18 & 21.98 \\
    tri2 & 43.77 & 28.56 & 32.07 & 21.98 \\
    tri3a & 41.23 & 25.81 & 29.62 & 19.50\\
    \textbf{tri4a} & \textbf{36.33} & \textbf{20.72} & \textbf{25.13} & \textbf{15.18}\\
    tri5a & 38.01 & 22.38 & 26.98 & 16.81\\
    \bottomrule
  \end{tabular}
  \vspace{0.5em}
  \centering \caption{使用新旧语言模型的最佳系统评分结果对比}
  \label{tab3}
\end{table}

\vspace{-1em}
由表\ref{tab3}可见,使用大语料后模型的WER和CER均有明显的下降,系统的性能进一步提高。相对而言,使用大语料后WER的下降比CER更加明显。

实际上,基于大语料训练得到的语言模型在训练过程tri3a阶段的WER和CER(分别为29.62\%和19.50\%)就已经优于baseline的最佳性能(tri4a,36.33\%和20.72\%)。

由此可见,使用大语料训练语言模型能很大程度地提升系统的性能,而且对WER影响相对更大。这一点也符合直觉。

\vspace{1em}
进一步对比两者最佳性能系统的解码结果。第一个代码块显示的是baseline(小语料)的结果,第二个代码块显示的是大语料训练后的结果,第三个代码块为标准的文本。

\begin{lstlisting}[language=bash]
BAC009S0904W0121 为了 解救 额度 化 问题 
BAC009S0904W0124 公积金 的 身影 贷款 直接 的 的 细察 王 
BAC009S0904W0128 广州 和 深圳 分 并未 五万 元 和 七万 元
\end{lstlisting}
\begin{lstlisting}[language=bash]
BAC009S0904W0121 为了 解决 入 额度 化 问题 
BAC009S0904W0124 公积金 的 商业 贷款 直接 的 弟媳 岔路 
BAC009S0904W0128 广州 和 深圳 分 别 为 五万 元 和 七万 元 \end{lstlisting}
\begin{lstlisting}[language=bash]
BAC009S0904W0121 为了 解决 额度 荒 的 问题 
BAC009S0904W0124 公积金 贷款 与 商业 贷款 之间 的 利息 差额 
BAC009S0904W0128 广州 和 深圳 分别 为 五万 元和 七万 元 \end{lstlisting}

大语料训练的系统在解码结果中相对更细致,如第一行的结果将“救”字错误地拆解为两个字(“决入”);但同时在准确率方面也更高,如第二行中“商业”的正确识别(baseline识别为“身影”)和第三行中“分别为”的正确识别(baseline识别为“分并未”)。


\section{训练DNN-HMM系统(15分)}

\setcounter{subsection}{-1}
\subsection{遇到的问题及解决}
调用$\tt{local/nnet3/run\_tdnn.sh}$脚本时,出现以下错误信息:

\begin{lstlisting}[language=bash]
Command 'python' not found, but can be installed with:
apt install python3
apt install python
apt install python-minimal
\end{lstlisting}

实际上,python已经成功安装,但python指令默认调用python 2.x版本。

\vspace{1em} \hspace{-2.2em}
\textbf{解决方案}:笔者找到了以下两种解决方案。

方案1. 为python3创建化名python,即输入以下指令。
\begin{lstlisting}[language=bash]
alias python=python3
\end{lstlisting}

方案2. 安装python-is-python3。


\subsection{训练流程}

DNN-HMM系统基于tri5a训练完毕后经fMMLR对齐的特征继续训练。主要流程分为ivector的准备和提取,及DNN部分的构建与训练。

\subsubsection{ivector的准备和提取}

\begin{enumerate}
    \setstretch{1.1}
    \item 使用不同的速度,基于训练集数据生成low-resolution speed-perturbed data;
    \item 生成3-way speed-perturbed 数据集train\_ sp;
    \item 在train\_sp数据上提取MFCC和pitch特征,并使用fMLLR进行对齐; 
    \item 整合tri5a对齐后的特征,对原train数据集进行音量扰动,计算MFCC和pitch特征后,分别对同时具有MFCC与pitch的特征和只有MFCC的特征使用CMVN归一化;
    \item 对开发集(dev)和测试集(test)也进行4中的处理;
    \item 对扰动后的数据集进行PCA降维和去相关,得到一组数据子集用于训练diagonal UBM(通用背景模型);
    \item 训练ivector提取器;
    \item 在train、dev、test数据集上提取ivector。
\end{enumerate}

\subsubsection{DNN部分的构建与训练}

\begin{enumerate}
    \setstretch{1.2}
    \item 生成DNN的结构文件network.xconfig,初始化神经网络。
    
    网络结构为8层,第一层为fixed affine layer,第二至七层为256个神经元的全连接层,每一层后有batch norm层和ReLU层,第八层为输出层。
    
    \item 训练神经网络。默认迭代训练42次,学习率自动调整。
    
    \item 使用训练得到的DNN-HMM系统对测试集test进行解码和评估。
\end{enumerate}

% 请简要介绍使用前述脚本训练DNN-HMM系统的流程,有关ivector的部分可以略过(可选:你遇到了什么问题?描述问题原因并介绍你的解决方法)。\textbf{写作完成后,请删除本段。}

\subsection{性能对比}

%在本节中,请将前述GMM-HMM系统及刚刚训练的DNN-HMM系统的性能列入表格以作对比。为使比较更加公平,应使用相同的语言模型。\textbf{写作完成后,请删除本段。}

笔者基于大语料的语言模型继续训练DNN-HMM系统。因此只对比基于大语料的GMM-HMM系统和DNN-HMM系统的评分结果。

\begin{table}[th]
  \centering
  \begin{tabular}{ c | c c }
    \toprule
    \textbf{系统} & \textbf{WER(\%)} & \textbf{CER(\%)} \\
    \midrule
    GMM-HMM & 25.13 & 15.18 \\
    DNN-HMM & 22.89 & 13.42 \\
    \bottomrule
  \end{tabular}
  \vspace{0.5em}
  \centering \caption{GMM-HMM与DNN-HMM评分结果对比}
  \label{tab2}
\end{table}

\vspace{-1em}
由上表可见,同样基于大语料训练得到的语言模型,使用DNN-HMM系统后的WER和CER均略低于GMM-HMM系统,性能相对更好。

\vspace{1em}
进一步对比两者最佳性能系统的解码结果。其中DNN-HMM系统的解码结果位于$\tt{exp/nnet3}$ $\tt{/new\_dnn\_sp/decode\_test/log}$目录下。

下面第一个代码块显示的是大语料GMM-HMM的结果,第二个代码块显示的是大语料DNN-HMM的结果,第三个代码块是标准文本。

\begin{lstlisting}[language=bash]
BAC009S0904W0121 为了 解决 入 额度 化 问题 
BAC009S0904W0124 公积金 的 商业 贷款 直接 的 弟媳 岔路 
BAC009S0904W0134 包括 仔仔 三 部委 发布 弘基 金 新政 
\end{lstlisting}
\begin{lstlisting}[language=bash]
BAC009S0904W0121 为了 给 就 额度 方 的 问题 
BAC009S0904W0124 公积金 贷款 商业 贷款 之间 的 弟媳 差额
BAC009S0904W0134 包括 次 三 部委 发布 公积金 新政 
\end{lstlisting}
\begin{lstlisting}[language=bash]
BAC009S0904W0121 为了 解决 额度 荒 的 问题 
BAC009S0904W0124 公积金 贷款 与 商业 贷款 之间 的 利息 差额 
BAC009S0904W0134 包括 此次 三 部委 发布 公积金 新政 \end{lstlisting}

可以发现DNN-HMM系统识别字词相对而言更加准确,如成功识别了第二行的“之间”和“差额”,第三行的“次”和“公积金”。同时,对语速不恒定的音频(如第一行的音频BAC009S0904W0121),DNN-HMM能捕捉时间较短的字词,如第一行中成功识别出了在音频中持续时间很短的“的”字,第二行中成功识别持续时间较短的“贷款”(在GMM-HMM系统中识别为“的”)。

\section{优化系统(20分)}

以下只是一些初步的想法,由于时间限制和资源限制,这些优化想法的具体实现并未完成。

\begin{enumerate}
\item RNN更适合序列的识别问题,相对DNN而言应该更适合语音转文本问题,而且本身也一定程度实现了HMM的功能。

因此,如果使用RNN替代DNN(即搭建RNN-HMM系统),应该能进一步取得比较好的效果。
    
\item 语言模型的改进。

实际上,语言模型也需要考虑上下文语境。因此如果使用RNN来建模语言模型,或许也能获得更准确的效果。
\end{enumerate}

\vspace{6em}

\section{最佳性能}

CER=13.42\%,WER=22.89\%

\section{特别鸣谢}
\textbf{王崇华}同学和\textbf{季弋琨}同学。

通过与上述两位同学的讨论,笔者完成了本文中对脚本代码的理解和解读,以及对SAT部分具体流程、功能的解释。

\begin{thebibliography}{}

\bibitem[1]{ML} AI2611,机器学习

\bibitem[2]{phone} Phoneme的相关概念以及Triphone, https://zhuanlan.zhihu.

com/p/322194937

\bibitem[3]{LDA} Feature and model-space transforms in Kaldi, http://kaldi-asr.org/doc/transform.html

\bibitem[4]{DNN} Context and chunk-size in the "nnet3" setup, http://kaldi-asr.org/doc/dnn3\_scripts\_context.html

\end{thebibliography}

\end{document}