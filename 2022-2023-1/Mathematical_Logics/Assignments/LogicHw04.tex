\documentclass{article}
\usepackage[utf8]{inputenc}
\usepackage{amsmath}
\usepackage{amsfonts}
\usepackage{setspace}
\usepackage{amsthm}
\usepackage{amssymb}
\usepackage{bbm}
\usepackage{geometry}
\usepackage{verbatim}
\usepackage{mathrsfs}
\usepackage{graphicx}
\usepackage{proof}
\usepackage[ruled,lined,commentsnumbered]{algorithm2e}

\geometry{left=3cm,right=3cm,top=2.25cm,bottom=2.25cm} 

\renewcommand{\qedsymbol}{\hfill $\blacksquare$\par}
\renewcommand{\emptyset}{\varnothing}
\renewcommand{\Pr}[2]{\mathbf{Pr}_{#1}\left[#2\right]}
\newcommand{\set}[1]{\left\{#1\right\}}
\newenvironment{solution}{\begin{proof}[\noindent\it Solution]}{\end{proof}}
\newcommand{\whiteqed}{\hfill $\square$\par}

\allowdisplaybreaks[4]

\setstretch{1.5}
\title{\textbf{Mathematical Logic Homework 04}}
\author{Qiu Yihang}
\date{Nov.8-10, 2022}

\begin{document}

\maketitle

\vspace{2.2em}
\section{DNF and CNF Formalization}
\vspace{1em}
\subsection{DNF Formalization}
\vspace{1em}
\begin{solution}
    The truth table of $\left(A\leftrightarrow B\right)\leftrightarrow C$ is as follows.

    \begin{table}[htbp]
        \centering
        \setstretch{1.2}
        \begin{tabular}{ccc|c}
            \hline
            $v(A)$ & $v(B)$ & $v(C)$ & $\bar{v}\left(\left(A\leftrightarrow B\right)\leftrightarrow C\right)$ \\
            \hline
            $\mathtt{True}$ & $\mathtt{True}$ & $\mathtt{True}$ & $\mathtt{True}$ \\
            $\mathtt{True}$ & $\mathtt{True}$ & $\mathtt{False}$ & $\mathtt{False}$ \\
            $\mathtt{True}$ & $\mathtt{False}$ & $\mathtt{True}$ & $\mathtt{False}$ \\
            $\mathtt{True}$ & $\mathtt{False}$ & $\mathtt{False}$ & $\mathtt{True}$ \\
            $\mathtt{False}$ & $\mathtt{True}$ & $\mathtt{True}$ & $\mathtt{False}$ \\
            $\mathtt{False}$ & $\mathtt{True}$ & $\mathtt{False}$ & $\mathtt{True}$ \\
            $\mathtt{False}$ & $\mathtt{False}$ & $\mathtt{True}$ & $\mathtt{True}$ \\
            $\mathtt{False}$ & $\mathtt{False}$ & $\mathtt{False}$ & $\mathtt{False}$ \\
            \hline
        \end{tabular}
    \end{table}

    \hspace{2.6em}
    Thus, the DNF of $\left(A\leftrightarrow B\right)\leftrightarrow C$ is 

    \vspace{-1.5em}
    $$\underline{\boldsymbol{\left(A\land B\land C\right)\lor\left(A\land\neg B\land\neg C\right)\lor\left(\neg A\land B\land C\right)\lor\left(\neg A\land\neg B\land\neg C\right)}.}$$

\vspace{-2.75em}
\end{solution}

\vspace{1em}
\subsection{CNF Formalization}
\vspace{1em}
\begin{solution}
    We can derive the CNF of $\left(A\leftrightarrow B\right)\leftrightarrow C$ from its DNF.

    \vspace{-2.5em}
    \begin{align*}
        & \left(A\land B\land C\right)\lor\left(A\land\neg B\land\neg C\right)\lor\left(\neg A\land B\land C\right)\lor\left(\neg A\land\neg B\land\neg C\right)\\
        |\!\!\!==\!\!\!|\ &\left(A\lor\left(A\land\neg B\land\neg C\right)\lor\left(\neg A\land B\land C\right)\lor\left(\neg A\land\neg B\land\neg C\right)\right)\\
        & \land \left(B\lor\left(A\land\neg B\land\neg C\right)\lor\left(\neg A\land B\land C\right)\lor\left(\neg A\land\neg B\land\neg C\right)\right) \\
        & \land \left(C\lor\left(A\land\neg B\land\neg C\right)\lor\left(\neg A\land B\land C\right)\lor\left(\neg A\land\neg B\land\neg C\right)\right) \\
        |\!\!\!= =\!\!\!|\ & \qquad \cdots \\
        |\!\!\!= =\!\!\!|\ & \left(A\lor A\lor\neg A\lor\neg A\right)\land\left(A\lor A\lor\neg A\lor\neg B\right)\land\left(A\lor A\lor\neg A\lor\neg C\right) \\
        & \land\left(A\lor A\lor B\lor\neg A\right)\land\left(A\lor A\lor B\lor\neg B\right)
        \lor\left(A\lor A\lor B\lor\neg C\right) \\
        & \land\left(A\lor A\lor C\lor\neg A\right)\land\left(A\lor A\lor C\lor\neg B\right)\land\left(A\lor A\lor C\lor\neg C\right) \\
        & \land\left(A\lor\neg B\lor\neg A\lor\neg A\right)\land\left(A\lor\neg B\lor\neg A\lor\neg B\right)\land\left(A\lor\neg B\lor\neg A\lor\neg C\right) \\
        & \land\left(A\lor\neg B\lor B\lor\neg A\right)\land\left(A\lor\neg B\lor B\lor\neg B\right)
        \lor\left(A\lor\neg B\lor B\lor\neg C\right) \\
        & \land\left(A\lor\neg B\lor C\lor\neg A\right)\land\left(A\lor\neg B\lor C\lor\neg B\right)\land\left(A\lor\neg B\lor C\lor\neg C\right) \\
        & \land\left(A\lor\neg C\lor\neg A\lor\neg A\right)\land\left(A\lor\neg C\lor\neg A\lor\neg B\right)\land\left(A\lor\neg C\lor\neg A\lor\neg C\right) \\
        & \land\left(A\lor\neg C\lor B\lor\neg A\right)\land\left(A\lor\neg C\lor B\lor\neg B\right)\land\left(A\lor\neg C\lor B\lor\neg C\right) \\
        & \land\left(A\lor\neg C\lor C\lor\neg A\right)\land\left(A\lor\neg C\lor C\lor\neg B\right)\land\left(A\lor\neg C\lor C\lor\neg C\right) \\
        & \land\left(B\lor A\lor\neg A\lor\neg A\right)\land\left(B\lor A\lor\neg A\lor\neg B\right)\land\left(B\lor A\lor\neg A\lor\neg C\right) \\
        & \land\left(B\lor A\lor B\lor\neg A\right)\land\left(B\lor A\lor B\lor\neg B\right)
        \lor\left(B\lor A\lor B\lor\neg C\right) \\
        & \land\left(B\lor A\lor C\lor\neg A\right)\land\left(B\lor A\lor C\lor\neg B\right)\land\left(B\lor A\lor C\lor\neg C\right) \\
        & \land\left(B\lor\neg B\lor\neg A\lor\neg A\right)\land\left(B\lor\neg B\lor\neg A\lor\neg B\right)\land\left(B\lor\neg B\lor\neg A\lor\neg C\right) \\
        & \land\left(B\lor\neg B\lor B\lor\neg A\right)\land\left(B\lor\neg B\lor B\lor\neg B\right)\land\left(B\lor\neg B\lor B\lor\neg C\right) \\
        & \land\left(B\lor\neg B\lor C\lor\neg A\right)\land\left(B\lor\neg B\lor C\lor\neg B\right)\land\left(B\lor\neg B\lor C\lor\neg C\right) \\
        & \land\left(B\lor\neg C\lor\neg A\lor\neg A\right)\land\left(B\lor\neg C\lor\neg A\lor\neg B\right)\land\left(B\lor\neg C\lor\neg A\lor\neg C\right) \\
        & \land\left(B\lor\neg C\lor B\lor\neg A\right)\land\left(B\lor\neg C\lor B\lor\neg B\right)\land\left(B\lor\neg C\lor B\lor\neg C\right) \\
        & \land\left(B\lor\neg C\lor C\lor\neg A\right)\land\left(B\lor\neg C\lor C\lor\neg B\right)\land\left(B\lor\neg C\lor C\lor\neg C\right) \\
        & \land\left(C\lor A\lor\neg A\lor\neg A\right)\land\left(C\lor A\lor\neg A\lor\neg B\right)\land\left(C\lor A\lor\neg A\lor\neg C\right) \\
        & \land\left(C\lor A\lor B\lor\neg A\right)\land\left(C\lor A\lor B\lor\neg B\right)
        \lor\left(C\lor A\lor B\lor\neg C\right) \\
        & \land\left(C\lor A\lor C\lor\neg A\right)\land\left(C\lor A\lor C\lor\neg B\right)\land\left(C\lor A\lor C\lor\neg C\right) \\
        & \land\left(C\lor\neg B\lor\neg A\lor\neg A\right)\land\left(C\lor\neg B\lor\neg A\lor\neg B\right)\land\left(C\lor\neg B\lor\neg A\lor\neg C\right) \\
        & \land\left(C\lor\neg B\lor B\lor\neg A\right)\land\left(C\lor\neg B\lor B\lor\neg B\right)\land\left(C\lor\neg B\lor B\lor\neg C\right) \\
        & \land\left(C\lor\neg B\lor C\lor\neg A\right)\land\left(C\lor\neg B\lor C\lor\neg B\right)\land\left(C\lor\neg B\lor C\lor\neg C\right) \\
        & \land\left(C\lor\neg C\lor\neg A\lor\neg A\right)\land\left(C\lor\neg C\lor\neg A\lor\neg B\right)\land\left(C\lor\neg C\lor\neg A\lor\neg C\right) \\
        & \land\left(C\lor\neg C\lor B\lor\neg A\right)\land\left(C\lor\neg C\lor B\lor\neg B\right)\land\left(C\lor\neg C\lor B\lor\neg C\right) \\
        & \land\left(C\lor\neg C\lor C\lor\neg A\right)\land\left(C\lor\neg C\lor C\lor\neg B\right)\land\left(C\lor\neg C\lor C\lor\neg C\right) \\
        |\!\!\!==\!\!\!|\ &\left(A\lor B\lor C\right)\land\left(A\lor\neg B\lor\neg C\right)\land\left(\neg A\lor\neg B\lor C\right)\land\left(\neg A\lor B\lor\neg C\right)
    \end{align*}

    Thus, the CNF of $\left(A\leftrightarrow B\right)\leftrightarrow C$ is 
    
    \vspace{-1.5em}
    $$\underline{\boldsymbol{\left(A\lor B\lor C\right)\land\left(A\lor\neg B\lor\neg C\right)\land\left(\neg A\lor\neg B\lor C\right)\land\left(\neg A\lor B\lor\neg C\right)}}$$

    \vspace{-2.7em}
\end{solution}

\vspace{1em}
\section{Problem 02}
\vspace{1em}
\begin{proof}
    By Compactness Theorem, we know if $\Delta'$ is finitely satisfiable, $\Delta'$ is satisfiable.

    \hspace{1.3em}
    Since $\Sigma\cup\set{\neg\alpha}$ is not satisfiable, we know $\Sigma\cup\set{\neg\alpha}$ is not finitely satisfiable.

    \hspace{1.3em}
    i.e. exists a finite subset $\Delta^*\subset\Sigma\cup\set{\neg\alpha}$ s.t. $\Delta^*$ is not satisfiable.

    \hspace{1.3em}
    \textbf{CASE 01}. $\neg\alpha\in\Delta^*$. Let $\Delta^*=\Delta\cup\set{\neg\alpha}$. Then $\Delta\subset\Sigma$. 
    
    \hspace{6.6em}
    We know $\Delta\cup\set{\neg\alpha}$ is not satisfiable $\Longleftrightarrow \Delta|\!\!\!=\alpha$.

    \hspace{1.3em}
    \textbf{CASE 02}. $\neg\alpha\notin\Delta^*$. Let $\Delta=\Delta^*\subset\Sigma$. Then $\Delta$ is not satisfiable. Thus, $\Delta|\!\!\!=\alpha$.

    \hspace{1.3em}
    In conclusion, there exists some finite set $\Delta$ s.t. $\Delta\subset\Sigma$.
\end{proof}

\vspace{1em}
\section{Semantic Consequences of $\boldsymbol{\Sigma}$ is Effectively Decidable}
\vspace{1em}
\begin{proof}
    Since $\Sigma$ is effectively enumerable, exists an algorithm $\mathcal{A}$ for enumerating members in $\Sigma$.
    
    \hspace{1.3em}
    We can design an algorithm $\mathcal{B}$ as follows.

    \vspace{-0.5em}
    \begin{algorithm}
        \setstretch{1.1}
        \SetKwProg{function}{\\Algo.}{begin}{end}
        \SetKwInOut{print}{Output}
        
	    \function{$\mathcal{B}$\\}
	    {
	    on \textbf{Input} $\alpha$\;
	    \For{$n=1,2,3,...$}
        {
            Run $\mathcal{A}$ until the $n$-th output appears\;
            Let the outputs of $\mathcal{A}$ be $\sigma_1,\sigma_2,...\sigma_n\in\Sigma$\;
            Collect all sentence symbols appearing in these wff in the set $S$\;
            Let $V\triangleq\set{f\mid f:S\to\set{\mathtt{True},\mathtt{False}}}$\;
            $not\gets 0$\;
            $yes\gets 0$\;
            \For{$v\in V$}
            {
                \lIf{$\bar{v}(\neg\alpha)=\mathtt{False}$}{$not\gets not+1$}
                \lIf{$\bar{v}(\alpha)=\mathtt{False}$}{$yes\gets yes+1$}
            }
            \lIf{$not==|V|$}{\qquad\textbf{Output:}"YES"}
            \lIf{$yes==|V|$}{\qquad\textbf{Output:}"NO"}
        }
        }
    \end{algorithm}

    \hspace{1.3em}
    Let the set of semantic consequences of $\Sigma$ be $\Gamma$.

    \hspace{1.3em}
    Now we prove $\mathcal{B}$ is an algorithm for effectively determining membership in $\Gamma$.

    \hspace{1.3em}
    \underline{First we prove the correctness}.

    \hspace{1.3em}
    If the algorithm returns "YES", then there exists a subset $\Delta=\set{\sigma_1,\sigma_2,...\sigma_n}\subset\Sigma$ s.t. for all truth assignment $v$ satisfying $\Delta$, $\bar{v}(\neg\alpha)=\mathtt{False} \Longrightarrow \Delta\nvDash\neg\alpha \Longrightarrow\Sigma\nvDash\neg\alpha$. Since for each wff $\alpha$, either $\Sigma\vDash\alpha$ or $\Sigma\vDash\neg\alpha$, we know $\Sigma\vDash\alpha$, i.e. $\alpha\in\Gamma$.

    \hspace{1.3em}
    If the algorithm returns "NO", then there exists a subset $\Delta=\set{\sigma_1,\sigma_2,...\sigma_n}\subset\Sigma$ s.t. for all truth assignment $v$ satisfying $\Delta$, $\bar{v}(\alpha)=\mathtt{False}\Longrightarrow\Delta\nvDash\alpha\Longrightarrow\Sigma\nvDash\alpha$, i.e. $\alpha\notin\Gamma$.

    \hspace{1.3em}
    Thus, the result of $\mathcal{B}$ is correct. \whiteqed

    \vspace{1em}\hspace{1.3em}
    \underline{Then we prove the algorithm will terminate within finite steps.}

    \hspace{1.3em}
    In the loop of $n$, since $S$ is finite, we know $\set{f\mid f:S\to\set{\mathtt{True},\mathtt{False}}}$ is finite. Calculating $\bar{v}(\neg\alpha)$ and $\bar{v}(\alpha)$ can terminate within finite steps. Thus, each loop will terminate within finite steps.
    
    \hspace{1.3em}
    If $\alpha\in\Gamma$, then exists a finite subset $\Delta\subset\Sigma$ s.t. $\Delta\vDash\alpha$. Obvious there must exist some $k$ s.t. $\Delta\subset\set{\sigma_1,\sigma_2...\sigma_k}$. Then exists a $v$ satisfying $\set{\sigma_1,...\sigma_k}$ s.t. $\bar{v}(\alpha)=\mathtt{True}$. Thus, $\mathcal{B}$ will terminate within $k$ loops.

    \hspace{1.3em}
    If $\alpha\notin\Gamma$, since either $\Sigma\vDash\alpha$ or $\Sigma\vDash\neg\alpha$, we know $\Sigma\vDash\neg\alpha\Longrightarrow\neg\alpha\in\Gamma$. Thus, $\mathcal{B}$ will still terminate within $k$ loops.

    \hspace{1.3em}
    Thus, $\mathcal{B}$ will return a result within finite steps. \whiteqed

    \vspace{1em} \hspace{1.3em}
    In conclusion, $\Gamma$ is effectively decidable, 
    
    \hspace{1.3em}
    i.e. semantic consequences of $\Sigma$ is effectively decidable.
\end{proof}

\vspace{1em}
\section{Proof Trees}
\vspace{1em}
\subsection{$\boldsymbol{A\land(B\lor C)\to (A\land B)\lor (A\land C)}$}
\vspace{1em}
\begin{solution}
    The proof tree is as follows.

    \vspace{-1.5em}
    \begin{align*}
        \infer[\to\text{-I}]
        {A\land(B\lor C)\to (A\land B)\lor (A\land C)}
        {\infer[\lor\text{-E}]
            {(A\land B)\lor (A\land C)}
            {
                \infer[\land\text{-E}]
                {B\lor C}
                {[A\land(B\lor C)]}
                &
                \quad
                \infer[\lor\text{-I1}]
                {(A\land B)\lor (A\land C)}
                {
                    \infer[\land\text{-I}]
                    {A\land B}
                    {
                        \infer[\land\text{-E1}]
                        {A}
                        {[A\land(B\lor C)]}
                        & \quad
                        [B]
                    }
                }
                &
                \infer[\lor\text{-I1}]
                {(A\land B)\lor (A\land C)}
                {
                    \infer[\land\text{-I}]
                    {A\land C}
                    {
                        \infer[\land\text{-E1}]
                        {A}
                        {[A\land(B\lor C)]}
                        & \quad
                        [C]
                    }
                }
            }
        }
    \end{align*}
    
\vspace{-3.6em}
\end{solution}

\vspace{1em}
\subsection{$\boldsymbol{\neg\left(\neg A\lor\neg B\right)\to A\lor B}$}
\vspace{1em}
\begin{solution}
    The proof tree is as follows.

    \vspace{-1.5em}
    \begin{align*}
        \infer[\to\text{-I}]
        {\neg\left(\neg A\lor\neg B\right)\to A\lor B}
        {
            \infer[\lor\text{-E}]
            {A\lor B}
            {
                \left\{B\lor\neg B\right\}
                &
                \infer[\lor\text{-I2}]
                {A\lor B}
                {[B]}
                &
                \quad
                \infer[\neg\text{-E}]
                {
                    A\lor B
                }
                {
                    \infer[\lor\text{-I2}]
                    {\neg A\lor\neg B}
                    {[\neg B]}
                    &
                    \quad
                    [\neg\left(\neg A\lor\neg B\right)]
                }
            }
        }
    \end{align*}

\vspace{-3.3em}
\end{solution}

\vspace{1em}
\subsection{$\boldsymbol{\left(\left(P\to Q\right)\to P\right)\to P}$}
\vspace{1em}
\begin{solution}
    The proof tree is as follows.

    \vspace{-1.5em}
    \begin{align*}
        \infer[\to\text{-I}]
        {\left(\left(P\to Q\right)\to P\right)\to P}
        {
            \infer[\lor\text{-E}]
            {P}
            {
                \left\{P\lor\neg P\right\}
                &
                [P]
                &
                \infer[\to\text{-E}]
                {P}
                {
                    \infer[\to\text{-I}]
                    {P\to Q}
                    {
                        \infer[\neg\text{-E}]
                        {Q}
                        {[P] & [\neg P]}
                    }
                    & 
                    [\left(P\to Q\right)\to P]
                }
            }
        }
    \end{align*}

\vspace{-3.3em}
\end{solution}

\vspace{1em}
\section{Part of the Proof of the Soundness of Natural Deduction}
\vspace{1em}
\subsection{$\boldsymbol{\neg}$-I Case}
\vspace{1em}
\begin{proof}
    We have $\set{\gamma_1,...\gamma_n,\beta}\vdash \delta\land\neg\delta$.

    \hspace{1.3em}
    By \textbf{Inductive Hypothesis}, $\set{\gamma_1,...\gamma_n,\beta}\vDash \delta\land\neg\delta$.

    \hspace{1.3em}
    i.e. for any truth assignment $v$ satisfying $\gamma_1,...\gamma_n,\beta$, $\bar{v}(\delta\land\neg\delta)=\mathtt{True}$. Thus, There exists no truth assignment $v$ satisfies $\gamma_1,...\gamma_n,\beta$.

    \hspace{1.3em}
    Therefore, $\set{\gamma_1,...\gamma_n,\beta}$ is not satisfiable. 
    
    \hspace{1.3em}
    Assume $\set{\gamma_1,...\gamma_n}\nvDash\neg\beta$. For any truth assignment $v$ satisfying $\set{\gamma_1,...\gamma_n}$, $\bar{v}(\neg\beta)=\mathtt{False}$, i.e. $\bar{v}(\beta)=\mathtt{True}$, i.e. exists $v$ satisfying $\set{\gamma_1,...\gamma_n}$ s.t. $v$ satisfies $\beta$, i.e. $v$ satisfies $\set{\gamma_1,...\gamma_n,\beta}$. \underline{\textbf{Contradiction.}}

    \hspace{1.3em}
    Thus, $\set{\gamma_1,...\gamma_n}\vDash\neg\beta$.

    \hspace{1.3em}
    Since $\set{\gamma_1,...\gamma_n}\subset\Sigma$, we know $\Sigma\vDash\neg\beta$.
\end{proof}

\vspace{1em}
\subsection{$\boldsymbol{\neg}$-E Case}
\vspace{1em}
\begin{proof}
    We know $\set{\gamma_1,...\gamma_n}\vdash\beta$ and $\set{\sigma_1,...\sigma_m}\vdash\neg\beta$.

    \hspace{1.3em}
    By \textbf{Inductive Hypothesis}, $\set{\gamma_1,...\gamma_n}\vDash\beta$ and $\set{\sigma_1,...\sigma_m}\vDash\neg\beta$.

    \hspace{1.3em}
    Since $\set{\gamma_1,...\gamma_n}\subset\Sigma,\set{\sigma_1,...\sigma_m}\subset\Sigma$, we know $\Sigma\vDash\beta$ and $\Sigma\vDash\neg\beta$.
    
    \hspace{1.3em}
    Thus, for any truth assignment $v$ satisfying $\Sigma$, $\bar{v}(\beta)=\bar{v}(\neg\beta)=\mathtt{True}$.

    \hspace{1.3em}
    Therefore, there exists no truth assignment $v$ satisfying $\Sigma$, i.e. $\Sigma$ is not satisfiable. 
    
    \hspace{1.3em}
    Then for any wff $\delta$, $\Sigma\vDash\delta$.

    \hspace{1.3em}
    Thus, $\Sigma\vDash\alpha$.
\end{proof}

\end{document}