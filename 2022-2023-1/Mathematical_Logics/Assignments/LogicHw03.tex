\documentclass{article}
\usepackage[utf8]{inputenc}
\usepackage{amsmath}
\usepackage{amsfonts}
\usepackage{setspace}
\usepackage{amsthm}
\usepackage{amssymb}
\usepackage{bbm}
\usepackage{geometry}
\usepackage{verbatim}
\usepackage{mathrsfs}
\usepackage{graphicx}
\usepackage{mathabx}
\usepackage[ruled,lined,commentsnumbered]{algorithm2e}

\geometry{left=3cm,right=3cm,top=2.25cm,bottom=2.25cm} 

\renewcommand{\qedsymbol}{\hfill $\blacksquare$\par}
\renewcommand{\emptyset}{\varnothing}
\renewcommand{\Pr}[2]{\mathbf{Pr}_{#1}\left[#2\right]}
\newcommand{\set}[1]{\left\{#1\right\}}
\newenvironment{solution}{\begin{proof}[\noindent\it Solution]}{\end{proof}}
\newcommand{\whiteqed}{\hfill $\square$\par}

\allowdisplaybreaks[4]

\setstretch{1.5}
\title{\textbf{Mathematical Logic Homework 03}}
\author{Qiu Yihang}
\date{Oct.21, 2022}

\begin{document}

\maketitle

\vspace{2.2em}
\section{The Binary Relation of Membership in Effectively Decidable Subset of $\boldsymbol{\mathbbm{N}}$ is Not Effectively Decidable}
\vspace{1em}
\begin{proof}
    Assume $B$ is effectively decidable.
    
    \hspace{1.3em}
    Then exists an algorithm $\mathcal{B}$ for determining the membership of input $(m,n)$ in $B$.

    \hspace{1.3em}
    Since $A_0,A_1,...,A_n,...$ is a listing of all effectively decidable sets of $\mathbbm{N}$, we know there exists algorithms $\mathcal{A}_0, \mathcal{A}_1,...,\mathcal{A}_n,...$ for determining the membership in $A_0,A_1,...,A_n,...$. 
    
    \hspace{1.3em}
    We can construct a subset $A^*$ of $\mathbbm{N}$ as follows. For any number $n\in\mathbbm{N}$, 

    \vspace{-1.2em}
    $$
    \left\{
        \begin{array}{rl}
            n\notin A^*, & \text{if algorithm }\mathcal{B}\text{ returns "YES" on input }(n,n) \\
            n\in A^*, & \text{if algorithm }\mathcal{B}\text{ returns "NO" on input }(n,n)
        \end{array}
    \right.
    $$

    \hspace{1.3em}
    Now we prove $A^*$ is an effectively decidable. We can construct the following algorithm $\mathcal{A}^*$.
    
    \vspace{-1em}
    \begin{algorithm}
        \setstretch{1.1}
        \SetKwProg{algo}{\\Algo.}{begin}{end}
        \SetKwInOut{print}{Output}
        
	    \algo{$\mathcal{A}^*$\\}
	    {
        on \textbf{Input} $n$\;
        Find $\mathcal{A}_n$ in the listing $\mathcal{A}_0,\mathcal{A}_1,...,\mathcal{A}_m,...$ and run algorithm $\mathcal{A}_n$ on $n$\;
        \leIf{the result is "YES"}
        {\quad\textbf{Output: }"NO"\quad}
        {\quad\textbf{Output: }"YES"}
        }
    \end{algorithm}

    \vspace{-1.5em}\hspace{6em}
    $n\in A^*\Leftrightarrow \mathcal{A}_n$ on input $n$ returns "NO" $\Leftrightarrow \mathcal{A}^*$ on input $n$ returns "YES".

    \hspace{6em}
    $n\notin A^*\Leftrightarrow \mathcal{A}_n$ on input $n$ returns "YES" $\Leftrightarrow \mathcal{A}^*$ on input $n$ returns "NO".

    \hspace{1.3em}
    Thus, $\mathcal{A}^*$ determines the membership in $A^*$, i.e. $A^*$ is an effectively decidable subset of $\mathbbm{N}$.

    \hspace{1.3em}
    Since $A_0,A_1,...,A_n,...$ is a listing of \underline{\textit{all}} effectively decidable subsets of $\mathbbm{N}$, exists $k\in\mathbbm{N}$ s.t. $A^*=A_k$. Consider the membership of $k$ in $A^*$.

    \hspace{1.3em}
    When $k\in A^*$, $k\in A^*\Rightarrow\mathcal{A}^*$ returns "YES" on $k\Rightarrow\mathcal{A}_k$ returns "NO" on $k\Rightarrow k\notin A_k\Rightarrow k\notin A^*.$

    \hspace{1.3em}
    When $k\notin A^*$, 
    $k\notin A^*\Rightarrow\mathcal{A}^*$ returns "NO" on $k\Rightarrow\mathcal{A}_k$ returns "YES" on $k\Rightarrow k\in A_k\Rightarrow k\in A^*.$
    
    \hspace{1.3em}
    \underline{\textbf{Contradiction.}}
    
    \hspace{1.3em}
    Thus, $B$ is \underline{\textbf{not}} effectively decidable.
\end{proof}

\section{For any Wff $\boldsymbol{\alpha}$, $\boldsymbol{s(\alpha) = 1+c(\alpha)}$}
\vspace{1em}
\begin{proof}
    Proof by induction.

    \hspace{1.3em}
    $S=\set{\alpha\mid\alpha\text{ is a wff and $s(\alpha)=1+c(\alpha)$.}}$

    \hspace{1.3em}
    Now we prove $S=$ the set of all wffs.

    \hspace{1.3em}
    \underline{\textbf{BASE CASE.}} $\alpha=A$, where $A$ is a sentence symbol. 
    
    \hspace{1.3em}
    $c(\alpha)=0, s(\alpha)=1.$ Thus, $s(\alpha)=1+c(\alpha)$ holds, i.e. $\alpha\in S$.

    \hspace{1.3em}
    \underline{\textbf{INDUCTIVE CASE.}}

    \hspace{1.3em}
    (1) $\alpha=(\neg\beta)$. Obvious $s(\alpha)=s(\beta), c(\alpha)=c(\beta)$. Assume $\beta\in S$.

    \hspace{7.7em}
    Then $s(\alpha)=s(\beta)=1+c(\beta)=1+c(\alpha)$. Thus, $\alpha\in S$. \whiteqed

    \hspace{1.3em}
    (2) $\alpha=(\beta\land\gamma)$. Then $s(\alpha)=s(\beta)+s(\gamma), c(\alpha)=c(\beta)+c(\gamma)+1$.

    \hspace{8.8em}
    Assume $\beta,\gamma\in S$.
    Then $s(\alpha)=s(\beta)+s(\gamma)=c(\beta)+1+c(\gamma)+1=1+c(\alpha)$.

    \hspace{8.8em}
    Thus, $\alpha\in S$. \whiteqed

    \hspace{1.3em}
    (3) Similarly, for cases where $\alpha=(\beta\lor\gamma)$ or $\alpha=(\beta\to\gamma)$ or $\alpha=(\beta\leftrightarrow\gamma)$, 

    \hspace{2.9em}
    when $\beta,\gamma\in S$, $s(\alpha)=s(\beta)+s(\gamma)=1+c(\beta)+1+c(\gamma)=1+c(\alpha)$.

    \hspace{2.9em}
    Thus, $\alpha\in S$. \whiteqed

    \vspace{1em} \hspace{1.3em}
    Therefore, $S=$ the set of all wffs.

    \hspace{1.3em}
    Thus, for any wff $\alpha$, $s(\alpha)=1+c(\alpha)$, i.e. the number of occurrences of sentence symbols in $\alpha$ is 1 greater than the number of binary connectives in $\alpha$.
\end{proof}

\vspace{1em}
\section{Parsing Tree of a Wff}
\vspace{1em}
\begin{solution}
    The resulting parse tree is as follows.

    \begin{figure}[htbp]
    	\centering
    	{\includegraphics[width=0.5\textwidth]{LogicHw03-3.pdf}}
        \caption{Parse Tree}
    \end{figure}

    \vspace{-0.5em} \hspace{2.6em}
    How the algorithm constructs the parse tree above is explained as follows.
    
    \hspace{2.6em}
    First create a single node $\left(\left(A\lor\left(B\land C\right)\right)\leftrightarrow\left(\left(A\lor B\right)\land\left(A\lor C\right)\right)\right)$.

    \hspace{2.6em}
    The only existing leaf node is not a sentence symbol. Since the second symbol is not $\neg$, 
    find 
    
    $\left(A\lor\left(B\land C\right)\right)$ followed by $\leftrightarrow$. The remaining part is $\left(\left(A\lor B\right)\land\left(A\lor C\right)\right)$. Create two children.

    \hspace{2.6em}
    Leaf node $\left(A\lor\left(B\land C\right)\right)$ is not a sentence symbol. Since the second symbol is not $\neg$, find $A$ 
    
    and $\left(B\land C\right)$. Create two children.

    \hspace{2.6em}
    Leaf node $\left(\left(A\lor B\right)\land\left(A\lor C\right)\right)$ is not a sentence symbol. Since the 2nd symbol is not $\neg$, 
    
    find $\left(A\lor B\right)$ and $\left(A\lor C\right)$. Create two children.

    \hspace{2.6em}
    Leaf node $A$ is a sentence symbol.

    \hspace{2.6em}
    Leaf node $\left(B\land C\right)$ is not a sentence symbol. Since the 2nd symbol is not $\neg$, find $B$ and $C$. 
    
    \hspace{2.6em}
    Create two children.

    \hspace{2.6em}
    Leaf node $\left(A\lor B\right)$ is not a sentence symbol. Since the 2nd symbol is not $\neg$, find $A$ and $B$. 
    
    \hspace{2.6em}
    Create two children.

    \hspace{2.6em}
    Leaf node $\left(A\lor C\right)$ is not a sentence symbol. Since the 2nd symbol is not $\neg$, find $A$ and $C$. 
    
    \hspace{2.6em}
    Create two children.

    \hspace{2.6em}
    All leaf nodes $A,B,C,A,B,A,C$ are sentence symbols. 
    
    \hspace{2.6em}
    Terminate.
\end{solution}

\vspace{1em}
\section{Problem 4}
\vspace{1em}
\subsection{Whether $\boldsymbol{(P\land Q)\to R\vDash(P\to R)\lor(Q\to R)}$ or not}
\vspace{1em}
\begin{solution}
    $(P\land Q)\to R\vDash(P\to R)\lor(Q\to R)$. The proof is as follows.

    \hspace{2.6em}
    Under truth assignment $v$, 

    \vspace{-2em}
    \begin{align*}
        &\bar{v}\big[(P\to R)\lor(Q\to R)\big]=\mathtt{False} \\
        \text{\textbf{only if.}}\quad &\bar{v}(P\to R)=\mathtt{False}\text{ and }\bar{v}(Q\to R)=\mathtt{False} \\
        \text{\textbf{only if.}}\quad &v(P)=v(Q)=\mathtt{True}, v(R)=\mathtt{False}.
    \end{align*}
    
    \hspace{2.6em}
    $v$ does not satisfy $((P\land Q)\to R)$, given that $\bar{v}\big[((P\land Q)\to R)\big]=\mathtt{False}$.

    \vspace{1em} \hspace{2.6em}
    Thus, truth assignments that do not satisfy $(P\to R)\lor(Q\to R)$ 
    
    \hspace{2.6em}
    \textbf{only if.} they do not satisfy $((P\land Q)\to R)$, 
    
    \hspace{5em}
    i.e. truth assignments satisfying $((P\land Q)\to R)$ satisfies $(P\to R)\lor(Q\to R)$. 

    \vspace{1em} \hspace{2.6em}
    Therefore, \underline{$\boldsymbol{(P\land Q)\to R}$ \textbf{tautologically implies} $\boldsymbol{(P\to R)\lor(Q\to R)}$}.
\end{solution}

\vspace{1em}
\subsection{Whether $\boldsymbol{(P\land Q)\to R\ |\!\!\!==\!\!\!|\ \vDash \Dashv (P\to R)\lor(Q\to R)}$ or not}
\vspace{1em}
\begin{solution}
    $(P\land Q)\to R\ |\!\!\!==\!\!\!|\ (P\to R)\lor(Q\to R)$. The proof is as follows.

    \hspace{2.6em}
    The truth table of $(P\land Q)\to R$ and $(P\to R)\lor(Q\to R)$ is as follows.

    \begin{table}[htbp]
        \centering
        \setstretch{1.2}
        \begin{tabular}{ccc|c|c}
            \hline
            $v(P)$ & $v(Q)$ & $v(R)$ & $\bar{v}((P\land Q)\to R)$ & $\bar{v}((P\to R)\lor(Q\to R))$ \\
            \hline
            $\mathtt{True}$ & $\mathtt{True}$ & $\mathtt{True}$ & $\mathtt{True}$ & $\mathtt{True}$ \\
            $\mathtt{True}$ & $\mathtt{True}$ & $\mathtt{False}$ & $\mathtt{False}$ & $\mathtt{False}$ \\
            $\mathtt{True}$ & $\mathtt{False}$ & $\mathtt{True}$ & $\mathtt{True}$ & $\mathtt{True}$ \\
            $\mathtt{True}$ & $\mathtt{False}$ & $\mathtt{False}$ & $\mathtt{True}$ & $\mathtt{True}$ \\
            $\mathtt{False}$ & $\mathtt{True}$ & $\mathtt{True}$ & $\mathtt{True}$ & $\mathtt{True}$ \\
            $\mathtt{False}$ & $\mathtt{True}$ & $\mathtt{False}$ & $\mathtt{True}$ & $\mathtt{True}$ \\
            $\mathtt{False}$ & $\mathtt{False}$ & $\mathtt{True}$ & $\mathtt{True}$ & $\mathtt{True}$ \\
            $\mathtt{False}$ & $\mathtt{False}$ & $\mathtt{False}$ & $\mathtt{True}$ & $\mathtt{True}$ \\
            \hline
        \end{tabular}
    \end{table}

    \vspace{-0.5em} \hspace{2.6em}
    Thus, under any truth assignment $v$, $\bar{v}\big[(P\land Q)\to R\big]=\bar{v}\big[(P\to R)\lor(Q\to R)\big]$.

    \hspace{2.6em}
    Therefore, any truth assignment satisfying $(P\land Q)\to R$ satisfies $(P\to R)\lor(Q\to R)$ while any truth assignment satisfying $(P\to R)\lor(Q\to R)$ satisfies $(P\land Q)\to R$.

    \vspace{1em} \hspace{2.6em}
    Thus, \underline{$\boldsymbol{(P\land Q)\to R}$ \textbf{is tautologically equivalent to} $\boldsymbol{(P\to R)\lor(Q\to R)}$.}
\end{solution}

\vspace{5em}
\section{Problem 5}
\vspace{1em}
\subsection{$\boldsymbol{((P\to Q)\to P)\to P}$ Is a Tautology}
\vspace{1em}
\begin{solution}
    $((P\to Q)\to P)\to P$ is a tautology. The proof is as follows.

    \hspace{2.6em}
    The truth table of $((P\to Q)\to P)\to P$ is as follows.

    \begin{table}[htbp]
        \centering
        \setstretch{1.2}
        \begin{tabular}{cc|cc|c}
            \hline
            $v(P)$ & $v(Q)$ & $\bar{v}[P\to Q]$ & $\bar{v}\big[(P\to Q)\to P\big]$ & $\bar{v}\big[((P\to Q)\to P)\to P\big]$\\
            \hline
            $\mathtt{True}$ & $\mathtt{True}$ & $\mathtt{True}$ & $\mathtt{True}$ & $\mathtt{True}$\\
            $\mathtt{True}$ & $\mathtt{False}$ & $\mathtt{False}$ & $\mathtt{True}$ & $\mathtt{True}$\\
            $\mathtt{False}$ & $\mathtt{True}$ & $\mathtt{True}$ & $\mathtt{False}$ & $\mathtt{True}$\\
            $\mathtt{False}$ & $\mathtt{False}$ & $\mathtt{True}$ & $\mathtt{False}$ & $\mathtt{True}$\\
            \hline
        \end{tabular}
    \end{table}

    \hspace{2.6em}
    Thus, for any truth assignment $v$, $\bar{v}\big[((P\to Q)\to P)\to P\big]=\mathtt{True}$.

    \hspace{2.6em}
    Therefore, $((P\to Q)\to P)\to P$ \underline{\textbf{is a tautology.}}
\end{solution}

\vspace{1em}
\subsection{$\boldsymbol{(A\leftrightarrow B)\to\neg((A\to B)\to\neg(B\to A))}$ Is a Tautology}
\vspace{1em}
\begin{solution}
    $(A\leftrightarrow B)\to\neg((A\to B)\to\neg(B\to A))P$ is a tautology. The proof is as follows.

    \hspace{2.6em}
    The truth table of $(A\leftrightarrow B)\to\neg((A\to B)\to\neg(B\to A))$ is as follows.

    \begin{table}[htbp]
        \centering
        \setstretch{1.2}
        \begin{tabular}{cc|c}
            \hline
            $v(A)$ & $v(B)$ & $\bar{v}\big[(A\leftrightarrow B)\to\neg((A\to B)\to\neg(B\to A))\big]$\\
            \hline
            $\mathtt{True}$ & $\mathtt{True}$ & $\mathtt{True}$\\
            $\mathtt{True}$ & $\mathtt{False}$ & $\mathtt{True}$\\
            $\mathtt{False}$ & $\mathtt{True}$ & $\mathtt{True}$\\
            $\mathtt{False}$ & $\mathtt{False}$ & $\mathtt{True}$\\
            \hline
        \end{tabular}
    \end{table}

    \hspace{2.6em}
    Thus, for any truth assignment $v$, $\bar{v}\big[(A\leftrightarrow B)\to\neg((A\to B)\to\neg(B\to A))\big]=\mathtt{True}$.

    \hspace{2.6em}
    Therefore, $(A\leftrightarrow B)\to\neg((A\to B)\to\neg(B\to A))$ \underline{\textbf{is a tautology.}}
\end{solution}

\vspace{3em}
\section{Problem 6}
\vspace{1em}
\subsection{If $\boldsymbol{\Sigma\vDash\alpha}$ Then For Any $\boldsymbol{\beta}$, $\boldsymbol{\Sigma\vDash\beta\to\alpha}$}
\vspace{1em}
\begin{proof}
    $\Sigma\vDash\alpha\Longrightarrow$ for any truth assignment $v$ satisfying $\Sigma$, $\bar{v}(\alpha)=\mathtt{True}$.

    \hspace{4em}
    $\Longrightarrow$ for any $\beta$, for any truth assignment $v$ satisfying $\Sigma$, $\bar{v}(\beta\to\alpha)=\mathtt{True}$. 
    
    \hspace{6em}
    ( Since when $\bar{v}(\alpha)=\mathtt{True}$, either $\bar{v}(\beta)=\mathtt{True}$ or $\mathtt{False}$, $\bar{v}(\beta\to\alpha)=\mathtt{True}$. )

    \hspace{4em}
    $\Longrightarrow$ for any $\beta$, for any truth assignment $v$ satisfying $\Sigma$, $v$ satisfies $\beta\to\alpha$. 

    \hspace{4em}
    $\Longrightarrow$ for any $\beta,\ \underline{\boldsymbol{\Sigma\vDash\beta\to\alpha}}$. 
\end{proof}

\vspace{1em}
\subsection{$\boldsymbol{\Sigma,\beta\vDash\alpha\text{ iff. }\Sigma\vDash\beta\to\alpha}$}
\vspace{1em}
\begin{proof}
    First we prove $\Sigma,\beta\vDash\alpha\Longrightarrow\Sigma\vDash\beta\to\alpha$.

    \hspace{1.3em}
    $\Sigma,\beta\vDash\alpha\Longrightarrow$ for any truth assignment $v$ satisfying $\Sigma$ and $\beta$, $\bar{v}(\alpha)=\mathtt{True}$.

    \hspace{5.1em}
    $\Longrightarrow$ for any truth assignment $v$ satisfying $\Sigma$ s.t. $\bar{v}(\beta)=\mathtt{True}$, we have $\bar{v}(\alpha)=\mathtt{True}$.

    \hspace{5.1em}
    $\Longrightarrow$ for any truth assignment $v$ satisfying $\Sigma$, $\bar{v}(\beta\to\alpha)=\mathtt{True}$.

    \hspace{7em}
    ( Since for truth assignment $v$ satisfying $\Sigma$,

    \hspace{7.6em}
    When $\bar{v}(\beta)=\mathtt{False},$ either $\bar{v}(\alpha)=\mathtt{True}$ or $\mathtt{False}$, $\bar{v}(\beta\to\alpha)=\mathtt{True}$.

    \hspace{7.6em}
    When $\bar{v}(\beta)=\mathtt{True}\Longrightarrow\bar{v}(\alpha)=\mathtt{True}$, we also have $\bar{v}(\beta\to\alpha)=\mathtt{True}$. )

    \hspace{5.1em}
    $\Longrightarrow$ for any truth assignment $v$ satisfying $\Sigma$, $v$ satisfies $\beta\to\alpha$.

    \hspace{5.1em}
    i.e. $\ \Sigma\vDash\beta\to\alpha$. \whiteqed

    \vspace{1em} \hspace{1.3em}
    Now we prove $\Sigma\vDash\beta\to\alpha\Longrightarrow\Sigma,\beta\vDash\alpha$.

    \hspace{1.3em}
    $\Sigma\vDash\beta\to\alpha\Longrightarrow$ for any truth assignment $v$ satisfying $\Sigma$, $\bar{v}(\beta\to\alpha)=\mathtt{True}$.

    \hspace{6.3em}
    $\Longrightarrow$ for any truth assignment $v$ satisfying $\Sigma$ s.t. $\bar{v}(\beta)=\mathtt{True}$, we have $\bar{v}(\alpha)=\mathtt{True}$.

    \hspace{8.2em}
    ( Otherwise, $\bar{v}(\beta\to\alpha)=\mathtt{False}$. )

    \hspace{6.3em}
    $\Longrightarrow$ for any truth assignment $v$ satisfying $\Sigma$ and $\beta$, $\bar{v}(\alpha)=\mathtt{True}$.

    \hspace{6.3em}
    i.e. $\ \Sigma,\beta\vDash\alpha$. \whiteqed

    \vspace{2em} \hspace{1.3em}
    Therefore, $\Sigma\vDash\beta\to\alpha$ \textbf{iff.} $\Sigma,\beta\vDash\alpha$.
\end{proof}

\end{document}