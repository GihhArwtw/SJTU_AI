\documentclass{article}
\usepackage[utf8]{inputenc}
\usepackage{amsmath}
\usepackage{amsfonts}
\usepackage{setspace}
\usepackage{amsthm}
\usepackage{amssymb}
\usepackage{bbm}
\usepackage{geometry}
\usepackage{verbatim}
\usepackage{mathrsfs}
\usepackage{graphicx}
\usepackage{proof}
\usepackage[ruled,lined,commentsnumbered]{algorithm2e}

\geometry{left=3cm,right=3cm,top=2.25cm,bottom=2.25cm} 

\renewcommand{\qedsymbol}{\hfill $\blacksquare$\par}
\renewcommand{\emptyset}{\varnothing}
\renewcommand{\Pr}[2]{\mathbf{Pr}_{#1}\left[#2\right]}
\newcommand{\set}[1]{\left\{#1\right\}}
\newenvironment{solution}{\begin{proof}[\noindent\it Solution]}{\end{proof}}
\newcommand{\whiteqed}{\hfill $\square$\par}

\allowdisplaybreaks[4]

\setstretch{1.5}
\title{\textbf{Mathematical Logic Homework 06}}
\author{Qiu Yihang}
\date{Dec.18, 2022}

\begin{document}

\maketitle

\vspace{3em}
\section{$\boldsymbol{\vDash_{\mathfrak{A}}\forall v_2Qv_1v_2[\![c^{\mathfrak{A}}]\!]\ \Longleftrightarrow\ \vDash_{\mathfrak{A}}\forall v_3Qcv_3}$}
\vspace{1em}
\begin{proof}
    We know

    \vspace{-3.em}
    \begin{align*}
    \vDash_{\mathfrak{A}}\forall v_2Qv_1v_2[\![c^{\mathfrak{A}}]\!]\  &\Longleftrightarrow\text{ for any }a\in|\mathfrak{A}|,\ \vDash_{\mathfrak{A}} Qv_1v_2[\![c^{\mathfrak{A}}, a]\!] \\
    & \Longleftrightarrow\ \text{for any }a\in|\mathfrak{A}|, \left(c^{\mathfrak{A}},a\right)\in Q^{\mathfrak{A}}. \\
    & \Longleftrightarrow\ \text{for any }a\in|\mathfrak{A}|, \vDash_\mathfrak{A} Qcv_3[\![a]\!]. \\
    & \Longleftrightarrow\ \ \vDash_{\mathfrak{A}}\forall v_3Qcv_3.
    \end{align*}

    \vspace{-3.3em}
\end{proof}

\vspace{1em}
\section{Wffs Defining Relations in $\boldsymbol{\mathfrak{A}}$}
\vspace{1em}
\subsection{$\boldsymbol{\set{0,1}}$}
\vspace{1em}
\begin{solution}
    We know

    \vspace{-3em}
    \begin{align*}
        a=0\ &\Longleftrightarrow\ \text{for any } b\in|\mathfrak{A}|, a\times b=a  \  \Longleftrightarrow\ \text{for any }b\in|\mathfrak{A}|,\ \vDash_\mathfrak{A} a\ \dot{\times}\ v_2\ \dot{=}\ a\ [\![b]\!] \ \  \\
        & \Longleftrightarrow\ \vDash_\mathfrak{A} \forall v_2 \left(v_1\ \dot{\times}\ v_2\ \dot{=}\ v_1\right)[\![a]\!]\ \Longleftrightarrow\quad\vDash_\mathfrak{A} \forall v_2 \left(v_1\ \dot{\times}\ v_2\ \dot{=}\ v_1\right)[\![a]\!]
    \end{align*}

    \vspace{-4.2em}
    \begin{align*}
        a=1\ &\Longleftrightarrow\ \text{for any } b\in|\mathfrak{A}|, a\times b = b \ \Longleftrightarrow\ \text{for any }b\in|\mathfrak{A}|,\ \vDash_\mathfrak{A} a\ \dot{\times}\ v_2\ \dot{=}\ v_2\ [\![b]\!] \\
        & \Longleftrightarrow\ \vDash_\mathfrak{A} \forall v_2 \left(v_1\ \dot{\times}\ v_2\ \dot{=}\ v_2\right)[\![a]\!]\  \Longleftrightarrow\quad\vDash_\mathfrak{A} \forall v_2 \left(v_1\ \dot{\times}\ v_2\ \dot{=}\ v_2\right)[\![a]\!]
    \end{align*}

    \vspace{-4.2em}
    \begin{align*}
        a = 0\text{ or }a = 1\ &\Longleftrightarrow\ \left(\vDash_\mathfrak{A} \forall v_2 \left(v_1\ \dot{\times}\ v_2\ \dot{=}\ v_1\right)[\![a]\!]\right) \text{or}\left(\vDash_\mathfrak{A} \forall v_2 \left(v_1\ \dot{\times}\ v_2\ \dot{=}\ v_2\right)[\![a]\!]\right) \\
        & \Longleftrightarrow\quad\vDash_\mathfrak{A}\left(\forall v_2 \left(v_1\ \dot{\times}\ v_2\ \dot{=}\ v_1\right)\right) \lor \left(\forall v_2 \left(v_1\ \dot{\times}\ v_2\ \dot{=}\ v_2\right)\right)[\![a]\!]
    \end{align*}

    \hspace{2.6em}
    Thus, $\boldsymbol{\left(\forall v_2 \left(v_1\ \dot{\times}\ v_2\ \dot{=}\ v_1\right)\right) \lor \left(\forall v_2 \left(v_1\ \dot{\times}\ v_2\ \dot{=}\ v_2\right)\right)}$ is a wff defining $\set{0,1}$.
\end{solution}

\vspace{1em}
\subsection{$\boldsymbol{\set{2}}$}
\vspace{.5em}
\begin{solution}
    Let $\varphi_{1}(x)=\forall v_2 \left(x\ \dot{\times}\ v_2\ \dot{=}\ v_2\right)$.
    
    \hspace{2.6em}
    From \textbf{2.1}, we know 
    $a=1\ \Longleftrightarrow\ \vDash_{\mathfrak{A}}\phi_1(v_1)[\![a]\!]$, i.e. $\varphi_1(x)$ defines $\set{1}$.

    \vspace{0.15em} \hspace{2.6em}
    Then

    \vspace{-5.1em}
    \begin{align*}
        a=2\ &\Longleftrightarrow\ \text{there is some }b\in|\mathfrak{A}|,\ a = b + b \text{ and }b = 1. \\
        & \Longleftrightarrow\ \text{there is some }b\in|\mathfrak{A}|,\ \vDash_\mathfrak{A} \left(a\ \dot{=}\ v_3\ \dot{+}\ v_3\land \varphi_1(v_3)\right)[\![b]\!]. \\
        & \Longleftrightarrow \quad \vDash_\mathfrak{A} \exists v_3 \left(v_1\ \dot{=}\ v_3\ \dot{+}\ v_3\land \varphi_1(v_3)\right)[\![a]\!].
    \end{align*}

    \vspace{-0.8em} \hspace{2.6em}
    Thus, $\boldsymbol{\exists v_3 \left(v_1\ \dot{=}\ v_3\ \dot{+}\ v_3\land \forall v_2 \left(v_3\ \dot{\times}\ v_2\ \dot{=}\ v_2\right)\right)}$ is a wff defining $\set{2}$.
\end{solution}

\vspace{1em}
\subsection{$\boldsymbol{\set{n\in\mathbb{N}\mid n\text{ is an even number}}}$}
\vspace{.5em}
\begin{solution}
    Let $\varphi_2(x)=\exists v_3 \left(x\ \dot{=}\ v_3\ \dot{+}\ v_3\land \varphi_1(v_3)\right)$.
    Then 

    \vspace{-2.8em}
    \begin{align*}
        a\text{ is an even number}\ &\Longleftrightarrow\ \text{there is some }b\in|\mathfrak{A}|=\mathbb{N},\ a = b + b. \\
        &\Longleftrightarrow\ \text{there is some }b\in|\mathfrak{A}|,\ \vDash_\mathfrak{A} \left(a\ \dot{=}\ v_2\ \dot{+}\ v_2\right)[\![b]\!]. \\
        &\Longleftrightarrow\ \vDash_\mathfrak{A} \exists v_2 \left(v_1\ \dot{=}\ v_2\ \dot{+}\ v_2\right) [\![a]\!].
    \end{align*}

    \vspace{-0.8em} \hspace{2.6em}
    Thus, $\boldsymbol{\exists v_2 \left(v_1\ \dot{=}\ v_2\ \dot{+}\ v_2\right)}$ is a wff defining $\set{n\in\mathbb{N}\mid n\text{ is an even number}}$.
\end{solution}

\vspace{1em}
\section{Homomorphism From $\boldsymbol{\mathfrak{N}_1}$ and $\boldsymbol{\mathfrak{N}_2}$}
\vspace{.5em}
\begin{proof}
    We can construct a function $h:\mathbb{N}\to\mathbb{N}$, $h(n)=2^n$.

    \hspace{1.3em}
    Now we prove that $h$ is a homomorphism from $\mathfrak{N}_1$ to $\mathfrak{N}_2$.

    \hspace{1.3em}
    \textbf{0.} Obvious $h$ is a function from $|\mathfrak{N}_1|=\mathbb{N}$ to $|\mathfrak{N}_2|=\mathbb{N}$.

    \hspace{1.3em}
    \textbf{1.} Since there is \underline{\textbf{no}} predicate symbol in $\mathbb{L}$, it is definite that 
    
    \hspace{5em}
    for any $n$-ary predicate symbol $R$ other than $\dot{=}$ and $a_1,...a_n\in|\mathfrak{N}_1|=\mathbb{N}$, 
    
    \vspace{-1.75em}
    $$(a_1,...a_n)\in R^{\mathfrak{N}_1}\ \Longleftrightarrow\ (h(a_1),...h(a_n))\in R^{\mathfrak{N}_2}.$$

    \vspace{-0.75em} \hspace{1.3em}
    \textbf{2.} There is only one function symbol in $\mathbb{L}$, i.e. $\dot{+}$.

    \hspace{2.7em}
    For any $a,b\in|\mathfrak{N}_1|=\mathbb{N}$, we have

    \vspace{-1.3em}
    $$\qquad h\left(\dot{+}^{\mathfrak{N}_1}(a,b)\right)=h(a+b)=2^{a+b}=2^a\times 2^b=h(a)\times h(b)=\dot{+}^{\mathfrak{N}_2}\left(h(a),h(b)\right).$$

    \vspace{-0.3em} \hspace{1.3em}
    \textbf{3.} There is only one constant symbol in $\mathbb{L}$, i.e. $\dot{0}$.

    \hspace{2.7em}
    We have $h\left(\dot{0}^{\mathfrak{N}_1}\right)=h(0)=2^0=1=\dot{0}^{\mathfrak{N}_2}$.

    \vspace{1.5em} \hspace{1.3em}
    Therefore, $h:|\mathfrak{N}_1|\to|\mathfrak{N}_2|,\ n\mapsto 2^n$ satisifes the properties of homomorphisms, 
    
    \hspace{1.3em}
    i.e. there is a homomorphism from $\mathfrak{N}_1$ to $\mathfrak{N}_2$.
\end{proof}

\end{document}
