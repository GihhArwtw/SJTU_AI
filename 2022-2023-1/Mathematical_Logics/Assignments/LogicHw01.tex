\documentclass{article}
\usepackage[utf8]{inputenc}
\usepackage{amsmath}
\usepackage{amsfonts}
\usepackage{setspace}
\usepackage{amsthm}
\usepackage{amssymb}
\usepackage{bbm}
\usepackage{geometry}
\usepackage{verbatim}
\usepackage{mathrsfs}
\usepackage{graphicx}
\geometry{left=3cm,right=3cm,top=2.25cm,bottom=2.25cm} 


\renewcommand{\qedsymbol}{\hfill $\blacksquare$\par}
\renewcommand{\emptyset}{\varnothing}
\renewcommand{\Pr}[2]{\mathbf{Pr}_{#1}\left[#2\right]}
\newcommand{\set}[1]{\left\{#1\right\}}
\newenvironment{solution}{\begin{proof}[\noindent\it Solution]}{\end{proof}}
\newcommand{\whiteqed}{\hfill $\square$\par}

\allowdisplaybreaks[4]

\setstretch{1.5}
\title{\textbf{Mathematical Logic Homework 01}}
\author{Qiu Yihang}
\date{Sept.20, 2022}

\begin{document}

\maketitle

\setcounter{section}{-1}
\section{Some Lemmas}

\vspace{1em}
\textbf{Thm.} The set $X$ is \textit{countable} \textbf{iff.} exists one-to-one mapping $f:X\to\mathbb{N}$. \quad [\textit{\underline{Already proved in class.}}]

\vspace{1em}\hspace{-1.8em}
\textbf{Thm.} When $f:B\to C$ is surjective and $g:A\to B$ is bijective, $f\circ g:A\to C$ is surjective.

\vspace{-0.5em}
\begin{proof}
    $f$ is surjective $\Rightarrow$ For any $c\in C$, we can always find some $b\in B$ s.t. $f(b)=c$.

    \hspace{1.3em}
    $g$ is bijective $\Rightarrow$ $g$ is surjective $\Rightarrow$ For any $b\in B$, we can always find some $a\in A$ s.t. $g(a)=b$.

    \hspace{1.1em}
    Thus, for any $c\in C$, we can always find some $a\in A$ s.t. $f\circ g(a) = f(g(a)) = c.$

    \hspace{1.3em}
    i.e. $f\circ g$ is surjective.    
\end{proof}

\vspace{3em}
\section{Question 01}
\vspace{1em}

\subsection{Domain and Range of $\boldsymbol{R}$}
\vspace{1em}
\begin{solution}
    By the definition of \textit{domain} and \textit{range}, we know
    
    \vspace{-3.em}
    \begin{align*}
        \mathtt{domain}(R) &= \set{1,2,3},\\\mathtt{range}(R) &= \set{1.1,3.2,2.0}
    \end{align*}

\vspace{-3.6em}
\end{solution}

\vspace{1em}
\subsection{$\boldsymbol{R}$ is Not a Function}
\vspace{1em}
\begin{solution}
    Since for $2\in B$, exist $1.1$ and $3.2$ s.t. $\langle2,1.1\rangle\in R$ and $\langle2,3.2\rangle\in R$, 
    
    \hspace{6em}
    by the definition of \textit{functions}, we know $R$ is \textbf{not} a function.
\end{solution}

\vspace{3em}
\section{Question 02}
\vspace{1em}
\subsection{$\boldsymbol{f:\mathbb{N}\rightarrow A}$ is Surjective $\boldsymbol{\Rightarrow A}$ is Countable}
\vspace{1em}
\begin{proof}
    When $f:\mathbb{N}\to A$ is surjective, for any $a\in A$, we know exists some $n\in\mathbb{N}$ s.t. $f(n)=a$.
    
    \hspace{1.3em}
    Then we can construct a function $g:A\to\mathbb{N}$ as follows.
    
    \hspace{3.9em}
    For any $a\in A$, we can pick a $n\in\mathbb{N}$ s.t. $f(n)=a$. Set $g(a)=n$.
    
    \hspace{1.3em}
    Now we prove $g$ is injective. 
    
    \vspace{-3.3em}
    \begin{align*}
        \qquad\qquad g(x)=g(y) \Rightarrow \ f(g(x))&=f(g(y)) \ \text{($f$ is a function.)} \Rightarrow x=y\ \text{(by the definition of $g$)}.
    \end{align*}
    
    \vspace{-1.5em} \hspace{1.3em}
    Thus, $g:A\to \mathbb{N}$ is a one-to-one mapping from $A$ to $\mathbb{N}$.
    
    \hspace{1.3em}
    Therefore, $A$ is countable.
\end{proof}

\subsection{$\boldsymbol{f:A\rightarrow\mathbb{N}}$ is Surjective $\boldsymbol{\Rightarrow A}$ is Infinite}
\vspace{1em}
\begin{proof}
    We prove it by contradiction.
    
    \hspace{1.3em}
    Assume $A$ is finite. Then exists a bijective $g:A\to \set{0,1,2,...,|A|-1}$. 
    
    \hspace{1.3em}
    When $f:A\to\mathbb{N}$ is surjetcive, we know for any $n\in\mathbb{N}$, we can find some $a\in A$ s.t. $f(a)=n$.
    
    \hspace{1.3em}
    Then $f\circ g^{-1}: \set{0,1,2,...|A|-1}\to\mathbb{N}$ is surjective, i.e. for any $n\in\mathbb{N}$, we can always find a $a\in\set{0,1,2,...|A|-1}$ s.t. $f\circ g^{-1}(a)=n.$
    
    \hspace{1.3em}
    Since $f$ and $g$ are functions, i.e. $f\circ g^{-1}:A\to\mathbb{N}$ is a function, we know $\mathtt{range}(f\circ g^{-1})$ is finite.

    \hspace{1.3em}
    Therefore, exists $n\in\mathbb{N}\setminus\mathtt{range}(f\circ g^{-1})$ s.t. for any $a\in\set{0,1,2,...|A|-1}, f\circ g^{-1}(a)\neq n.$ \underline{\textbf{Contradiction!}}
    
    \hspace{1.3em}
    Thus, $A$ is not finite, i.e. $A$ is infinite.
\end{proof}

\vspace{1em}
\section{Question 03}
\vspace{1em}
\begin{proof}
    We can construct a listing without repetitions as follows.

    \begin{figure}[htbp]
    	\centering
    	{\includegraphics[width=0.6\textwidth]{LogicHw01-3.pdf}}
    \end{figure}

    \vspace{-1em} \hspace{1.3em}
    Thus, $\mathbb{N}\times\mathbb{N}$ is enumerable.
\end{proof}
    

\end{document}