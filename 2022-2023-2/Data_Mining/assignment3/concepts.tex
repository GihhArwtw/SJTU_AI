\documentclass{article}
\usepackage[utf8]{inputenc}
\usepackage{amsmath}
\usepackage{amsfonts}
\usepackage{setspace}
\usepackage{amsthm}
\usepackage{amssymb}
\usepackage{geometry}
\usepackage{mathrsfs}
\geometry{left=3cm,right=3cm,top=2.25cm,bottom=2.25cm} 
\usepackage{graphicx}
\usepackage[ruled,lined,commentsnumbered]{algorithm2e}
\usepackage{bbm}
\usepackage{subfigure}
\usepackage{tikz}
\usepackage{booktabs}

\renewcommand{\qedsymbol}{\hfill $\blacksquare$\par}
\newcommand{\whiteqed}{\hfill $\square$\par}
\newcommand{\set}[1]{\left\{#1\right\}}
\newenvironment{solution}{\begin{proof}[\noindent\it Solution]}{\end{proof}}
\newenvironment{disproof}{\begin{proof}[\noindent\it Disproof]}{\end{proof}}
\newcommand{\staExp}[2]{\mathbb{E}_{#1}\left[#2\right]}
\renewcommand{\Pr}[2]{\mathbf{Pr}_{#1}\left[#2\right]}
\newcommand{\bd}[1]{\boldsymbol{#1}}

\allowdisplaybreaks[4]
\setstretch{1.5}


\title{\textbf{Data Mining Homework 03}}
\author{Qiu Yihang}
\date{May 2023}

\begin{document}


\maketitle

\vspace{3em}
\section{$\bd{\#}$(Independent Hash Functions) in Bloom Filter}
\vspace{1em}
\begin{solution}
    The optimal $k$ should give a minimal false positive probability.

    \hspace{2.6em}
    The fraction of false positive $1$ on a certain bit in the vector $B$ is $\left(1-e^{-\frac{km}{n}}\right)$.

    \hspace{2.6em}
    We only get a false positive result when all $k$ functions gives a false positive. 
    
    \hspace{2.6em}
    Thus, the probability for false positive is 
    
    \vspace{-2em}
    \begin{align*}
        \Pr{}{\mathtt{false\ positive}} & = \left(1-e^{-\frac{km}{n}}\right)^k \\
        \min\ \Pr{}{\mathtt{false\ positive}}\  &\Longleftrightarrow\ \frac{\partial}{\partial k}\left(1-e^{-\frac{km}{n}}\right)^k=0
    \end{align*}

    \hspace{2.6em}
    We have

    \vspace{-2em}
    \begin{align*}
        \qquad\qquad\frac{\partial}{\partial k}\left(1-e^{-\frac{km}{n}}\right)^k=0 &\Longleftrightarrow\ \left(1-e^{-\frac{km}{n}}\right)^k\left[\ln\left(1-e^{-\frac{km}{n}}+\frac{m}{n}\frac{k}{e^{\frac{km}{n}-1}}\right)\right]=0 \\
        &\Longleftrightarrow\ \text{either } 1-e^{-\frac{km}{n}}=0\text{ or } \ln\left(1-e^{-\frac{km}{n}}+\frac{m}{n}\frac{k}{e^{\frac{km}{n}-1}}\right) = 0\\
        &\Longleftrightarrow\  k = 0\text{ (\textit{discarded}) or }\frac{n}{m}\ln 2 
    \end{align*}

    \hspace{2.6em}
    Therefore, the optimal $k$ is $\dfrac{n}{m}\ln 2$.
\end{solution}

\vspace{1em}
\section{Moments}
\vspace{1em}
\begin{solution}
    The frequencies of values for stream $3,1,4,1,3,4,2,1,2$ are as follows.

    \begin{table}[htbp]
        \centering
        \begin{tabular}{c|cccc}
            \toprule
            Value & 1 & 2 & 3 & 4 \\
            Frequency & 3 & 2 & 2 & 2 \\
            \bottomrule
        \end{tabular}
    \end{table}

    
    Thus, 
    
    \hspace{2.6em}
    the second moment, i.e. the surprise number, is $3^2+2^2+2^2+2^2=21$, and
    
    \hspace{2.6em}
    the third moment is $3^3+2^3+2^3+2^3=51$.
\end{solution}

\vspace{1em}
\section{Problem 03}
\vspace{1em}

\textbf{(a).} The key attribute should be (\underline{the item purchased, the purchase price}). Sample a decent amount of samples for each item, and calculate the average price for them.

\hspace{-1.8em}
\textbf{(b).} The key attribute should be (\underline{the customer's ID, the purchase price.}) We should sample randomly under a uniform distribution.

\hspace{-1.8em}
\textbf{(c).} The key attribute should be (\underline{the item purchased, the customer's ID.}) We should sample randomly under a uniform distribution.

\end{document}
